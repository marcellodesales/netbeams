\documentclass[conference]{IEEEtran}

\usepackage{epsfig}
\usepackage{fancyvrb}

\DefineVerbatimEnvironment%
  {code}{Verbatim}{numbers=left,numbersep=3pt,frame=lines,%
                   xleftmargin=7pt,fontsize=\footnotesize}




\begin{document}


\title{An OSGi-based Sensor Network for \\ Environmental Monitoring}

\author{\authorblockN{XXX}
\authorblockA{San Francisco State University \\
Computer Science Department \\
1600 Holloway Avenue \\
San Francisco, CA 94132 \\
EMail: XXX@sfsu.edu}}


\maketitle
\begin{abstract}
  Abstract.
\end{abstract}


\section{Introduction}

The introduction should emphasize that we interpret sensor networks in
a different way: our focus is on off-the-shelf, highly specialized
sensors for environmental monitoring. Our focus is end-to-end: how to
collect data from a remote location with no stationary power or
Internet connection and deliver the data in near-realtime to a web
browser. Length: first page (including title and abstract).

\section{Use Case}

Start with some related work. Who else has done an OSGi-based
infrastructure for sensor networks? Kleber: I believe you have a few
references. This section should be a combination of a use case (using
the RTC and their YSI sonde as an example) and a requirements
analysis: requirements such as remote management and monitoring
capabilities, time-delayed communication, etc. In this section should
be no mentioned of DSP or OSGi. Length: 1 page.

\section{DSP}

Should describe the OSGi-based NetBEAMS architecture. Total length of
this section: 2 pages.

\subsection{Data Sensor Framework and Platform}

should include a picture of a standalone DSP showing a DC and a DP
(general description only). Explain difference between DSP Framework
and DSP Platform. This section should give a top-level overview.
Subsequent sub-sections should explain certain features of the DSP in
more detail.

\subsection{DP and DC}

Should give some API sniplets for DP and DC.

\subsection{DSP Message}

Should explain the way DSP Messages are defined: XML Schema and JAXB.

\subsection{Matcher}


\section{NetBEAMS}

Should explain some specific DC and DP we have implemented for
NetBEAMS. Length: 1 - 1.5 pages.

\subsection{YSI Sonde DP}

\subsection{Wire transport DC/DP}

\subsection{Web Managements}


\section{Conclusions and Outlook}

Length (including bibliography): 0.5 pages.

%\bibliography{../literature/lit}
%\bibliographystyle{plain}


\end{document}
