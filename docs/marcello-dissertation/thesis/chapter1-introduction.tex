% main.tex, to be used with thesis.tex
% This contains the main work of your thesis.



%\bibliography{thesis}  % uses the references stored in Chapter1Radar.bib

\chapter{Introduction}

Sensor networks are commonly used in different areas for different purposes
such as accurate measurements during scientific research \cite{sn-intro01} and
the general public access such as weather forecast \cite{sn-intro02}. In the
scientific community, they serve as tools to watch the state of the environment
by storing samples of data at particular periods of time. For example, the
NASA Jet Propulsion Laboratory uses the Volcano SensorWeb \cite{sn-ex02}, a
sensor network which aims at collecting data from individual volcanoes in
Alaska and South Pacific to be used with heuristics regarding hazardous
activities. Similarly, the National Data Buoy Center, a division of the
National Oceanic and Atmospheric Administration (NOAA) \cite{sn-ex03},
provides online information from different buoys in different shores around
the world regarding water quality, current information, etc, what
characterizes sensor network for environmental monitoring \cite{sn-ex01}.
Therefore, sensor networks can be seen as an extremely valuable tool
for the scientific and commercial sectors, whose interest is the use of
the sensor's produced data.

The first motivation for the inception of this work was to provide a data
storage system for NetBEAMS \cite{netbeams2009}, a component-based sensor
network infrastructure that can be used to automate the process of data
gathering in sensor networks. In this way, the review of the state of the
art in rewiewed that different approaches can be taken to provide access to the
collected sensor data: interrogate sensor devices indirectly by accessing its
data in places called network data sink or simply connect directly to the
device. While the collected data is only available in-memory as a data stream
in the latter approach, it is stored in in any sort of secondary memory
device such as a hard-disk or a flash memory for archival purposes in the
former approach. Along with the location of the collected data, the data model 
may also influence in the process of reusing the collected data. In this way,
the introduction of a persistence layer for existing sensor networks requires
an understanding about its infrastructure, the nature and the location of the
collected data is temporalily located after it is produced from the sensor
device. Similarly related to the infrastructure, the data model used to
represent the collected data is another aspect of the data persistence layer
implemented to manage the raw data values, and may rise questions. Considering
that sensor networks devices may change over time, how can a data model take
into account future additions of new entities to the database schema? Should 
such a solution take into account the end users' skills in database system or
provide access to the data through programming languages
\cite{sn-programming-language}? These questions are faced when the solution
considers the primary users of the collected data such as Biologists from the
the department of Biology at San Francisco State University, which are
responsible to maintain SF-BEAMS \cite{sfbeams2006}, a marine sensor network
used by NetBEAMS. While access to the collected data may represent
difficulties, other technical issues may also be faced such as growth in terms
of the number of sensor devices and, consequently, the number of data
produced. In this way, sensot networks engineered with the purpose of data
archival has to answer questions related to how disk storage can be maximized
when storing the collected data. Last, but not least, the way to read the data
for reuse is the most striking features of a sensor network, taking into
account the different data formats used by the primary users of the system.
How can individual collected sensor data be exported to other formats such as
spreadsheets in order to be shared among research colleagues? For these and
other reasons, the inception of a persistence storage for NetBEAMS, and any
other sensor network, can be an exceptionally challenging problem to solve.

In view of the fact that the persistence storage of a sensor network imposes
many different challenges, the assessment of the current state of the art for
data persistence in sensor networks revealed different alternatives to better
evaluate database technologies for sensor networks and, specifically speaking,
to NetBEAMS. As a matter of fact, this first contribution of this work is the
creation of a set of basic taxonomies for data persistence in sensor networks,
proposed to drive an empirical analysis of different database systems used by
researchers. In this way, based on the findings of the review of current trends
in database systems \cite{db-is-rdbs-dommed, db-shard-intro} and distributed
systems such as Cloud Computing \cite{cloud-comp-survey}, this work proposes a
data persistence solution based on mongoDB \cite{mongodb}, a schema-less
document-oriented database system. This selection was based after an empirical
analysis of the different characteristics of the relational
\cite{relational-model} and the key-value pair \cite{db-kvp} data models.
After designing and implementing a solution whose technology has not yet been
used in the sensor network community, experiments were conducted to simulate
the workload of real-world scenarios used by SF-BEAMS during one year of data
produced by a specific type of sensor. In conclusion, the implementation
attends most of the requirements and specifications defined by the taxonomies,
giving an alternative data mode that solves the problem related to constant
schema changes, providing an outstanding support for Data Provenance
\cite{sn-provenance} for sensor networks. In addition, the solution provides
different data access alternatives such as the use of programming languages or
REST Web Services \cite{http-rest} instead of the Structure Query Language SQL
\cite{sql} used in relational databases \cite{relational-model}, as well as
native support to export data to formats used by the SF-BEAMS staff.

This paper is organized as follows: the next chapter describes the literature
review as the state of art on sensor networks and data persistence, providing
the underlying foundation for the design of taxonomies in data persistence for
sensor networks in chapter 3. Then, chapter 4 examines the selected case study
and its requirements for a persistence layer, whose technology's properties are
evaluated against the developed taxonomies in chapter 5. The culmination of
this work is chapter 6, which details the design and implementation of a data
persistence solution for the selected case study. Likewise, chapter 7 discusses
the solution by presenting the measurements and results of the conducted
experiments. Last, but not least, chapter 8 presents the conclusions of this
work, together with the different possible future works not addressed by this
dissertation.
