% main.tex, to be used with thesis.tex
% This contains the main work of your thesis.


%\bibliography{thesis}  % uses the references stored in Chapter1Radar.bib

\chapter{Introduction}

Sensor networks are commonly used in different areas for different purposes
such as accurate measurements during scientific research \cite{sn-intro01} and
the general public access such as weather forecast \cite{sn-intro02}. In the
scientific community, they serve as tools to watch the state of the environment
by storing samples of data at particular periods of time. For example, the
NASA Jet Propulsion Laboratory uses the Volcano SensorWeb \cite{sn-ex02}, a
sensor network which aims at collecting data from individual volcanoes in
Alaska and South Pacific to be used with heuristics regarding hazardous
activities. Similarly, the National Data Buoy Center, a division of the
National Oceanic and Atmospheric Administration (NOAA) \cite{sn-ex03},
provides online information from different buoys in different shores around
the world regarding water quality, current information, etc, what
characterizes sensor network for environmental monitoring \cite{sn-ex01}.
Therefore, sensor networks can be seen as an extremely valuable tool
for the scientific and commercial sectors, makeing direct use of the sensor
networks' produced data.

In order to provide access to the collected sensor data, different approaches
can be taken: interrogate sensor devices indirectly by accessing its data in
places called network data sink or simply connect directly to the device. The 
While the latter approach the data is only available in memory as a data
stream, the former approach contains any sort of secondary memory such as a
hard-disk or a flash memory, which are used to archive the collected data.
Along with the location of the collected data from sensors, the the data model
and may also influence in the process of reusing the collected data. In this
way, the lack of a persistence layer for existing sensor networks requires a
deep understanding about its infrastructure, the nature and location of the
collected data, the chosen data model representation, among others. For
instance, how can a data persistence layer safely represent the collected
sensor devices' properties without the necessity of restructuring any existing
data model? Should the end users' skills in database system be taken into
account or the use of programming languages to access the data
\cite{sn-programming-language}? This problem is currently faced by the
department of Biology at San Francisco State University with a marine sensor
network managed by researchers without Computer Science skills. While the data
access may enforce the difficulties, technical issues may also increase while
the implementation of a sensor network such as growth in terms of the number
of sensor devices and, consequently, the number of data produced. Last, but
not least, the way to read the data for reuse is the most striking features of
the sensor network, taking into account the different data formats used by the
primary users of the system. How can individual collected sensor data be
exported to other formats such as spreadsheets in order to be shared among
research peers? For these reasons, the inception of a persistence storage can
be an exceptionally challenging problem to solve.

In view of the fact that the persistence storage of a sensor network imposes
many different challenges, this work proposes the assessment of the current
state of the art for sensor networks in order to provide a data persistence
layer for an existing sensor network. As a matter of fact, this work's first
contribution is the creation of a set of basic taxonomies for data persistence
in sensor networks, proposed to drive an empirical analysis of different
technologies used in the sensor network community. Furthermore, it also takes a
look at different alternatives not yet explored through experiments for a
selected sensor network as a case study. Similarly, this work makes another
novice contribution by suggesting a ``lighter" data model and database
infrastructure that simply captures sensor network devices' observations,
taking into account the non-predictable scalability of its infrastructure, as
well as aspects of the nature of the data such as identity, time and location
of the collected data.

This paper is organized as follows: the next chapter describes the literature
review as the state of art on sensor networks and data persistence, providing
the underlying foundation for the design of taxonomies in data persistence for
sensor networks in chapter 3. Then, chapter 4 examines the selected case study
and its requirements for a persistence layer, whose technology properties are
evaluated in chapter 5 against the developed taxonomies. The culmination of
this work is chapter 6, which details the design and implementation of a data
persistence solution for the selected case study. Likewise, chapter 7 discusses
the solution by presenting the measurements and conducted experiments, and
describes the findings and contributions of this work. Last, but not least,
chapter 8 presents the conclusions of this work, together with the possible
future works not addressed here.
