% main.tex, to be used with thesis.tex
% This contains the main work of your thesis.


%\bibliography{thesis}  % uses the references stored in Chapter1Radar.bib

\chapter{Introduction}

Sensor networks are commonly used in different areas for different purposes
such as scientific measurements and general population use
\cite{sn-intro01} \cite{sn-intro02}. In the scientific community, they serve as
tools to monitor the state of the environment by storing samples of data in specific 
periods of time. For instance, the NASA Jet Propulsion Laboratory uses the 
Volcano SensorWeb \cite{sn-ex02}, a sensor network which aims to collect data
from specific volcanoes in Alaska and South Pacific to be used with heuristics 
regarding dangerous activities. Similarly, the National Data Buoy Center, a
department of the National Oceanic and Atmospheric Administration (NOAA)
\cite{sn-ex03}, provides online data from different buoys in different coast
shores around the world regarding water quality, temperature, tides, etc, what 
characterizes sensor network for environmental monitoring \cite{sn-ex01}.
Therefore, sensor networks can be seen as a very important scientific research and 
commercial field with direct application and use of its produced data.

In order to provide access to the collected sensor data, different approaches
can be taken to interrogate sensor devices. In general, the query process can
take place on the network nodes themselves, as it is called in-network, or in 
a centralized data sink. The former is generally used when the data is
collected at the sensor node on its local storage, being shown by \cite{sn-storage01}
\cite{sn-storage03} \cite{sn-storage04} that this strategy helps mitigating energy
consumption. However, this strategy is used when data archival is not the
primary reason, given the limited resources on the sensor devices. For this
reason, the the latter strategy has been used when the collected data is
primarily archived in a centralized database system for later reuse
\cite{sn-storage02}.

When it comes to data model implementation, the relational data model
\cite{relational-model} is one of the most used approaches to represent data in
sensor network databases in any of the query processing strategies, and in most cases 
with a modified version of the SQL language \cite{sn-db-newop01}
\cite{sn-db-newop02}. However, the design of the database system must be done
with prior knowledge of the sensor types when  choosing the relational data
model, as the database design must be normalized \cite{db-normalization} for the entities
already identified for the system. For instance, the introduction of a new
sensor device might represent a potential change on the database structure by
seeking a new normalized version of the database structure. In addition to
challenges of maintaining the database schema normalized, the use of Data
Provenance approaches in sensor network data \cite{sn-provenance01} usually
addresses the problem related with the lineage of data  and its full history.
However, once the data model is in place to receive the collected data, one of
the most important non-functional requirements in sensor networks is
scalability of the persistence storage. Since an operational sensor network
can produce huge amounts of data, it must be able to cope with the increase
and data load during data gathering, and yet provide users the same
performance during data retrieval by using different strategies such a
data-centric storage \cite{sn-storage03}.

In summary, this dissertation discusses the solution of a centralized
data-centric persistence storage component for the NetBEAMS \cite{netbeams09},
Data Sensor Platform (DSP), an open-source \cite{open-source} implementation of
a sensor network that primarily focused on environmental data produced by NetBEAMS. 
Different challenges can be related to the DSP in the context of sensor 
networks: data model, data persistence organization and strategy, and user data
access and distribution. First, NetBEAMS needs to provide means to safely 
represent sensor devices' properties without the need of restructuring any 
existing data model, taking into account the use cases from the main users. 
For example, a marine biologist may query the the persistence storage regarding
the temperature of a specific geographic region at the San Francisco Bay at a 
given time frame. Second, since a sensor network may experiment grown in terms 
of number of sensors, and consequently the number of data produced, the
persistence  storage must be able to reasonably scale and be well structured
in terms of storage space with the least service interruption. For instance,
if the persistence storage is close to reach its storage capacity, what
should the network administrator do? Last, but not least, the means to access 
the data for reuse is the most important features of the system, taking into 
account the different data formats used by the main users of the system. How 
to export specific collected data to a spreadsheet or archive files to be 
published or shared with other users? For these, and many other reasons, the 
persistence storage problem addressed by this work can be seen as a very 
challenging experience to solve.

In view of the fact that the persistence storage of a sensor network imposes
many different challenges, and taking into account the case study NetBEAMS, this
work's main strategy focused analyzing the different specifications of data
persistence in the literature in order to find new types of mechanisms to
provide data persistence for sensor network data. In this way, the kernel of
this dissertation is focused on providing solutions to the real-world problem
face by the Romberg Tiburon Center and its use of NetBEAMS. Furthermore,
the evaluation of different technologies is discussed in order to provide a
loosely-coupled component-based solution based on the infrastructure of the 
DSP Platform. For the reason of providing a 'lighter' data model, a subtype of
the Key-Value Pair \cite{db-kvp01} data model was chosen to easily capture the
description of any sensor device's properties, along with the use of time
series data models \cite{sn-provenance01} for time snapshots. In that way, data
modeling process does not represent a new refactoring each time a new sensor 
type is added or removed from the persistence system, giving users means to 
retrieve data accordingly. Similarly, in contemplation of scalability, this 
work proposes the use of an increasingly popular infrastructure technique called
Database Sharding \cite{db-shard01} \cite{db-shard02}, which takes the approach
of Data-Centric storage \cite{sn-storage03} on in-network query processing in a
centralized database system. In order to provide such mechanism, this work
evaluates the use of shard keys based on the geographic origin of the collected
data as cited to be one of the options when working with Provenance data in 
sensor networks \cite{sn-provenance01}. As a result, the data persistence system
component was conceived using the plug-and-play capability from the DSP 
platform without doing  any changes to the current DSP Platform, providing an
scalable solution that can adjust according to any data load increase. Finally,
the selected technology provides different data access mechanisms, which
directly addressed problems related to data access and Information Fusion 
\cite{sn-info-fusion}, providing export-import capabilities and visualization through
the availability of APIs written in different popular programming languages.

This dissertation report is organized as follows: the next chapter is an 
overview of the state of art on topics that supports the implementation of the
solution of this work, providing the necessary background foundation for the 
following chapters. Chapter 3 presents the design and architecture of the 
conceived persistence module, describing the requirements needed to be
addressed,  while Chapter 4 describes the implementation based on the design. 
The analysis and investigation of the implemented solution is shown through 
the analyzes of the experiment results on Chapter 5. Last, but not least, 
Chapter 6 presents the conclusions of this work, followed by the possible 
future works and recommendations presented on Chapter 7.