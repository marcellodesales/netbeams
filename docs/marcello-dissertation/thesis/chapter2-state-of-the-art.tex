% main.tex, to be used with thesis.tex
% This contains the main work of your thesis.

%\bibliography{thesis}  % uses the references stored in Chapter1Radar.bib

\chapter{State of the Art of Data Persistence in Sensor Networks}

In general, this chapter reviews surveys and related papers on sensor networks,
providing a basic introduction about their characteristics to a new audience.
Similarly, it reviews the characteristics of related works that discuss and
provide data persistence solutions for sensor networks, characterizing the
different approaches used to described and model the gathered data from sensor
devices.

%It is important to note that the scope of this literature review is limited to
%our case study described in Chapter 4 and, therefore, it only
%covers a subset of disciplines in sensor networks and data persistence.

\section{Survey in Sensor Networks and Its Infrastructure}

According to \cite{sn-intro02}, sensor networks are specialized types of
network systems comprised of nodes with specific goals such as to produce and
collect data. The former is seen as a sensor device capable of producing data
to the network based on an observation from the environment. On the other
hand, the latter is described as the final destination of the raw data in a
sensor network and called the network data sink, or simply data sink, whose
responsibility is to receive, categorize and store the ``sensed'' data. In
this way, the location of the data sink is determined by the purpose of
the collected data: data nodes are placed in-network close to the sensor
devices when collected data has a life cycle, being discarded after being
utilized. In contrast, data nodes are placed in a centralized node when the
collected data is used for historical purposes.

The following sections describe general properties of sensor networks regarding
infrastructure and components as described in the literature.

\subsection{Properties of Sensor Device Nodes in Sensor Networks}

Each sensor node performs different tasks for the network and, therefore,
influences the method of deployment used. \cite{sn-intro01} identifies that
the deployment of sensor devices can use arbitrary methods such as being
dropped from an airplane, or manually placed at chosen geographic locations.
Moreover, the cited authors also mentioned that the deployment can be done as a
one-time event or as an iterative process, where the sensor devices are moved
to different places as needed.

As noted by \cite{sn-intro01} sensor nodes can be
moved around the deployment location, or remain in one place as a stationary sensor. 
In the case of a sensor network comprised with different
moving sensors, a given sensor node can move occasionally or continuously, as
a result of an accidental side-effect or a desired property of the device. As
a matter of fact, these sensors are automotive or they can be attached to
another moving object. For this reason, \cite{sn-intro01} classifies those
types of mobility as active or passive.

Considering the physical design and capabilities of sensor devices, they range
from microscopic to devices measuring a few cubic feet
\cite{sn-intro02}. In addition to their limited size, their computational
resources and power availability can also be affected, when they are located in
areas of limited access to energy. In this way, \cite{sn-intro01} also
distinguishes how sensor devices are usually designed to either store energy in
batteries or use energy scavenged from solar cells. Even though a sensor
network is designed for a purpose, \cite{sn-intro01} also explains that it can
be comprised of different sensor devices playing different roles in what they
observe in the environment, characterizing the heterogeneity of the sensor
network by the hardware capabilities and what types of data each of them
produce.

Finally, \cite{sn-intro01} mentions that sensor device nodes can communicate
with each other using different media such as radio, diffuse light or
inductive sound, influencing the type of protocols they use to exchange data. They
can have knowledge of their physical location if they are integrated with a
Global Positioning System (GPS) device, which gives the latitude or
longitude coordinate values. Therefore, the cost of these devices may range
from hundreds to thousands of dollars, depending on the design capabilities and functionality of a given device.

\subsection{Properties of the Infrastructure of Sensor Networks}
\label{sec:sn-infrastructure}

The last section discussed the different properties of the sensor devices when
placed in a sensor network. Depending on how they communicate with each other,
\cite{sn-intro02} states that sensor devices can be organized in a
structured, or ad hoc way. For this reason, the infrastructure of sensor
networks differ in topology such as single-hop, star, tree or  graph. Each of
these topologies may influence how the produced data from sensor nodes is
collected and routed to their related data sink. While there are only two
levels of node connectivity in sensor networks deployed as single-hop and
star topologies, sensor networks whose topology is either tree or graph use
multi-hop communication among each of their nodes for data communication from
the devices to the data sink.

The property of coverage on a sensor network is defined by
\cite{sn-intro01} as the physical space ``visible" to the sensors devices, which
is also directly related to the purpose of the network, its size in terms of
the number of nodes deployed and the sensor network lifetime. Therefore,
sensors are deployed in sparse or dense manners, according to the
specification of what needs to be covered by the sensor devices. Furthermore,
considering that some types of sensor devices are very likely to experience regular
hardware failures, \cite{sn-intro02} described sensor network examples with 
redundant nodes that provide better data coverage.
Finally, the coverage of sensor devices also has influence on the connectivity
property of the sensor nodes. In order to communicate with redundant nodes, a
physical connection must exist at all times. On the other hand, lower
level of coverage is related to sporadic communication among the sensor nodes.

The sensor network size is related to the degree of coverage mentioned earlier,
and it is measured by the number of devices deployed. \cite{sn-intro01}
reports examples of sensor networks composed by different ranges of deployed
sensor devices such as dozens, hundreds or even thousands of nodes. However,
the lifetime of a sensor network determines how long these devices are going
to be used, as well as how long the sensor network may exist. For example, an
environmental sensor network \cite{sn-ex01} with the purpose of collecting data
for historic purposes may have their sensor devices continuously performing
their data collection tasks. On the other hand, the lifetime of a
sensor network designed to achieve a specific goal will not exist after the sensor
devices finish their tasks. Finally, \cite{sn-intro02} argues that the
performance related to the orchestration of sensor devices in sensor networks
is related to the constraints defined for quality of service (QoS).  This includes
real-time data delivery within a threshold or for robustness such as
maintaining live data even though a communication link is not available.

All things considered, sensor nodes play a vital role during the execution
of a sensor network, whose infrastructure is characterized by how and where the
sensor nodes are deployed and communicate with each other. The next
section covers the approaches dealing with the problem of this type of system
network.

\section{Persistence Storage for Sensor Networks}
\label{sec:sn-persitence-storage}

As outlined in the last few sections, sensor networks are comprised of
different types of sensor nodes. An important sensor node type is the
data sink because it is where the raw data produced by sensor
nodes resides. Data collection and access in sensor networks is directly
related to the characteristics of the sensor network, its routing mechanisms
and where the data sink is placed within the network.

The type of query mechanism used to directly or indirectly interrogate sensor
devices in a sensor network is directly related to the aforementioned locations
of the collected data. The in-network query processing is used when the
location of the collected data is a local storage site \cite{sn-storage01,
sn-storage03} within the network; a strategy typically used to maximize the
sensor's resources, while minimizing energy consumption \cite{sn-storage04}. On
the contrary, the a typical centralized query processing is used to search for
values over the collected data when the location is a centralized data
node \cite{sn-storage02}.

When it comes to the data representation, the relational data model
\cite{relational-model} is one of the most traditional approaches used to
describe the collected data of sensor networks, as it is reported in database
systems such as the TinyDB \cite{sn-db-tinydb}. For instance, \cite{sn-db-newop}
proposes a modified version of the SQL language of TinyDB proposed to solve
SQL Join problems, so that the query processing takes into account the location
of the distributed data. Conversely, \cite{sn-provenance} proposes the use of
Data Provenance as the primary guidelines used to describe collected  raw data from
sensor devices.

The following sections describe general properties of sensor networks regarding
data persistence, as reviewed in the literature.

\subsection{Data Use and Where Data is Stored}
\label{sec:sn-data-purpose}
\label{sec:sn-storage-locations}

As presented in the previous section, the raw data is the one in-transit from
the sensor to the storage device, where it is labeled to be
identifiable. It is also referred to as the collected data. 
Initially, the raw data serves as the main outcome of the sensor network.
\cite{sn-provenance} differentiates the use of \textbf{real-time data stream}, when they are generated from a sensor
device whenever the data is requested. In this way, the data is still in its raw
format specified by the sensor manufacturer. However, it is not retained. 
In contrast, the authors describe the data as \textbf{archivable}, when they a
have historical significance and will be processed to be reused by other systems at
a latter time. While the real-time data streams are used and potentially
discarded, the archivable data is stored into secondary memory such as a
hard-disk.

After producing the ``sensed'' values for the observed parameters from the
environment, the sensor node is responsible for sending the raw data to a specific
data node defined by the network infrastructure, as delineated in section
\ref{sec:sn-infrastructure}. Thus, \cite{sn-storage03}
describes that the destination for the collected sensor data will be one of two
different types of storage devices: the \textbf{external storage} and the
\textbf{local storage}. In the former type, the sensor sends the
produced data to another sensor node that contains an external storage device.
In the latter the data is stored only within the memory of the device before its 
reading as a real-time data stream. On the whole, the sensor network
infrastructure dictates where the movement performed by the sensor data,
for example, \cite{sn-storage02} sees a wireless sensor networks as a many-to-one data
collection pattern, when its organization is based on single-hop or star. On
the other hand, data may also flow from one intermediate sensor node to another
until it reaches its final data sink when the data organization is structured
as a graph, tree \cite{sn-storage01, sn-storage03}.

Lastly, the use of different external storage devices can be used
to partition the collected data. This practice is called \textbf{Data-Centric
Storage}, and aims at clustering the collected data from different sensor
devices based on a given a property of the data such as location or time of
creation.

\subsection{Query Processing in Sensor Networks}
\label{sec:query-process}

The query processing in sensor networks is a direct or indirect method of
accessing the collected data stored at data nodes, whose location is determined
by different facts described in the previous section. \cite{sn-storage03}
describes an example of an \textbf{in-network query processing} when sensor
nodes, acting as data nodes, provides real-time data stream as a response
to queries requesting the device's collected data. Likewise, if the collected
data is in use for a specific period of time, the data is cached in an external
storage device in a close-by data node. In this way, when a query is issued to
the sensor network with these characteristics, the query processor considers 
all data nodes before it returns its final response to the query. In comparison, a sensor
network whose topology is in the form of a star, the data sink is usually
located in the center of the structure, and thus, receives all
the collected data from other sensor devices within the network. For this reason, it 
can be seen as a \textbf{centralized query processing}. Factoring in that a centralized data
sink receives all the collected data from the peer nodes, this query processing
strategy is responsible for responding to indirect queries about the
produced data from the network. 

Based on the literature review, there are advantages and disadvantages related to
each of the query processing approaches described with regards to the
infrastructure of the sensor network. \cite{sn-storage03} argues that the 
primary purpose of the in-network query processing is to minimize the amount
of data transmitted from sensor devices to a data sink, as sensor devices are
generally limited by energy use. Considering that sensor devices are also used
as data nodes, \cite{sn-storage04} states that this approach can potentially
flood all sensor nodes in the network, thereby decreasing its performance.
For this reason, \cite{sn-storage01} reports the use of the data-centric
storage by organizing the collected data based on its particular property
values. In this way, the query processor hits specific data nodes that contain
the types of data in the range of the requested data concluding that this approach 
decreases the network congestion and maximizes the
capacity on the sensor devices and the performance of the sensor network.

Unlike the in-network, the centralized query processing is related to a regular
database system, located in one of the network nodes. For this reason, this
indirect type of query processing is used when the data purpose of the data is for 
data storage. As a consequence, \cite{sn-storage02} shows that the result of
a many-to-one communication between the centralized data sink
and the other data nodes, this query process does not affect any of the sensor
nodes in the network and, thus, maximizes the sensor device utilization. 
Moreover, this strategy better exposes the collected data to the primary users
raw data, which follows the process of data processing, compression,
aggregation, and so on \cite{sn-db-modeling02}.

Although the approach of centralized query processing maximizes the sensor
nodes capacity and minimize energy consumption, \cite{sn-storage02} explains
that this trade off imposes the creation of the unavoidable development of a
point of traffic concentration or the phenomenon of network congestion. For
this reason, \cite{sn-storage04} describes that different techniques used in 
distributed systems such as the use of master-slave data replication or 
database partitioning \cite{db-partitioning-relational} can be used to reduce
or mitigate the effects of this funneling.

\section{Data Models and Query Engines for Sensor Networks}
\label{sec:data-models}

Different approaches are used to represent the collected data from sensors.
Some research groups lacking database experience arbitrarily assign data 
descriptors, while others use database technology in a more pragmatic approach.
\cite{sn-programming-language} describes several applications using
various types of languages for accessing the data designed in different data
models. \cite{sn-provenance} reports on the use of a \textbf{Tabular Data Model}
\cite{tabular-model} using comma-separated values (CSV) to organize the
measurements collected from sensor devices. In addition, \cite{sn-db-tinydb} 
describes the use of the \textbf{Relational Data Model} \cite{relational-model} 
as used in the majority of the projects that involves data persistence due to its
popularity in areas such as Software Engineering and Distributed Systems. Finally,
\cite{sn-xml-usage01, sn-xml-usage02} discusses examples of sensor network 
applications that make use of XML \cite{xml} to represent the collected data on a 
\textbf{Structured Data Model}.

The following sections describe the application of each of aforenamed data
models in the context of sensor networks.

\subsection{Tabular Data Model and Query}

The most basic computer application to process data is through the use of
spreadsheets, where data is organized into cells \cite{tabular-model}. When
exported to the so-called comma-separated values (CSV) file format, the
resulting file is comprised of lines of ASCII characters, separated by a
delimiter. The first line represents a header, with the identification of
columns of specific data, where the following lines contain the values
related to each top column. \cite{sn-provenance} reports that the tabular
data model is commonly used to provide data persistence for sensor networks.
The SF-BEAMS environmental sensor networks \cite{sfbeams2006} uses this
approach to distribute the collected data over the Internet using the OPenDAP
\cite{opendap} data format.

This data model can be used in different contexts, as \cite{sn-provenance}
describes that the versions of the same document can be easily managed and
shared among research colleagues using a version-control system such as
Subversion \cite{subversion}.

\subsection{Relational Data Model and its Query Mechanism}

\cite{sn-db-newop} shows the use of the Relational Data Model
\cite{relational-model} as it is used in sensor networks to describe the
collected data from sensor devices. The project described
by the authors is based on the TinyDB database system \cite{sn-db-tinydb}, a
database system specifically built for data nodes of sensor networks. It uses
the same properties and constraints of regular database systems, where the
entities of the sensor network are required to be modeled and represented by
tables.

In order to query the data collected from sensor devices, the use of the 
Structured Query Language (SQL) \cite{sql} is the main language used to extract
the data managed by the system. As mentioned earlier, the default versions of
this language have been modified to give support to the constraints of the
location of the database such as the addition of new SQL operators for TinyDB
\cite{sn-db-newop}.

\subsection{Structured Data Model and its Query Mechanism}

XML is an increasing popular data format used widely by the applications developed
for the Internet. \cite{sn-xml-usage01} reports an efficient way to represent
the collected data from sensors using XML, while the authors of
\cite{sn-xml-usage02} describe a sensor network using XML for data model
representation. In general, the collected data is structured into classes of
relating elements and validated by an XML Schema \cite{xml-schema}.

In order to extract data from XML-based databases, the XML XPath language
\cite{xml-xpath} is often used to query XML data. However, with the inception
of hybrid XML-relational database, \cite{sn-xml-usage03} reported the use XML
as the data model and the use of SQL to query the stored data on the IBM's DB2
database system. 

Other examples of the use of the structured data model using XML is found in
the use of middleware, as the authors of \cite{sn-xml-query-engines} reports.
Such an approach is used for data exchange by the implementing of a middleware
capable of query and extract the collected data from sensor devices
\cite{sn-xml-middleware}. A practical example of the use of XML as a middleware
to communicate between nodes is the communication mechanism used by NetBEAMS
\cite{netbeams2009}. See section the documentation at
\cite{netbeams-dsp-architecture} for details.

\section{Description of the Collected Data}
\label{sec:sn-data-description}

As described in the previous section, the collected data from sensor networks
can be represented by different data models, requiring the knowledge of the
properties of the data. Not to mention that the low-level or raw data produced
by a given sensor device might not include vital details regarding the
origin and identity of the collected data. For this reason, the collected data
is reported to be described in different ways such as Data Provenance or
Annotations.

\cite{sn-provenance} describes Data Provenance as a technique used to describe
those essential properties of the collected data from sensors. One typical
example of the application of Data Provenance is the so-called black-box of an
aircraft, which carries the collection of spatial-temporal data structures
collected from a group of sensors placed in different locations of an
aircraft, as shown by \cite{sn-exemple-blackbox}. Finally, the author describes 
the collected data as used in the process of data recovery and analysis in order to
identify the probable reason of an aircraft accident, by analyzing the time, 
location of the aircraft, and other independent variables from the sensors.

In general, sensor devices produce and preserve data onto their local
storage or transmit them to data nodes as described earlier in this chapter. It
is vital to understand that the majority of these devices output the raw data
in streams of values with any type of delimitation between each other. For
this reason, the processes of parsing and labeling each of the values located
in the data stream regularly requires the use of the device's specification
documentation. Similarly, in order to have access to each of the properties of
collected data by a given property related to time or space,
\cite{sn-provenance} also mentions that Metadata\footnote{Data about data}
should be included together with the collected data.

An example of a Data Provenance application is the use of the aggregation of data
over time to estimate the changing effects of the weather in a given region.
For this reason, the collected data from sensor devices need meaningful
annotations to better describe the identity of the collected data, as 
\cite{sn-provenance} reports. Pertinent questions regarding the
data can be asked such as \textbf{what was collected?}, \textbf{When was the
data collected?} or \textbf{Where was the data collected?}. Similarly, while
Data Provenance provides basic mechanisms to describe sensor networks data in
general, others see the importance of the time and position of sensor networks.
\cite{sn-time-series} describes strategies used to cluster distributed sensor
data as time series of data, as well as to decrease the size of data streams
being transmitted in real-time sensor networks when these collected data do
not need to be transferred \cite{sn-data-reduction}.

A different way to describe sensor devices is through the use of
Annotations. \cite{sn-annotation} shows the use of tags to describe data
collected from a physical area monitoring. The sensor network was
designed to monitor moving objects, capturing their location while in
movement and their visual appearance using a video camera. In this way,
adding tags to particular video frames can describe different them, and
therefore, providing an efficient way to search for a specific frame based on
its visual observations. Similarly, the above referenced project also uses data
provenance to describe its collected data, since each of the collected data 
contains the related GPS coordinates of the observed object.

Finally, data provenance provides practical use of the collected data from
different sensor devices and different sensor networks in the process called
Information Fusison \cite{sn-info-fusion}. \cite{sn-geo-metadata} is an example of
the aggregation of different data from different devices located in different 
geographic sensor networks which is capable of processing spatial-temporal 
queries of real-time or historical data. The following sections describe each 
of the metadata properties as defined by \cite{sn-provenance}, giving different
examples of each of them.

\subsection{Data as an object}

Different sensor devices produce different sets of attributes based on their
capabilities and design. In general, these data sets do not have distinct
names, which lead to difficulties when indices over the properties of the
collected data are required to make them searchable. Specially for the purpose
of data archival, the collected data needs to be identified, giving a meaning
of independence on its own, the same idea behind the unique identification of
entities on the relational data model \cite{relational-model}. Equally,
\cite{sn-provenance} describes that traditional sensor networks use the
collected data for analysis long after the attributes have been collected,
usually before it is categorized and indexed. Annotations and tags are also
names given by \cite{sn-provenance} as means to describe collected sensor data.

Briefly, the collected data is described by the name of the observation
property. For instance, \textbf{temperature} = 73.45 expresses the value of the
observation of the temperature. In other words, the string ``temperature'' is
the property describing the raw collected data 73.45. On the other hand,
the example \textbf{temperature-scale} = ``Fahrenheit" describes the metadata
regarding the temperature.

\subsection{Time of Data Creation}

In order to use the collected data for the purpose of historical analysis, the
use of a point in time is the most prominent feature of the data. Also referred to
as spatial-temporal data, it is considered a pivotal metadata used in different
types of sensor networks such as an aircraft the black-box
\cite{sn-exemple-blackbox}, environmental sensor network \cite{sfbeams2006},
and data communications management systems \cite{sn-dataware-house}. In this
way, Other projects classify data in time-series \cite{sn-time-series-example},
where data is related to all the observations a given sensor device performed
over a period of time. The apparent implications is that all the produced data
is necessary, but \cite{sn-time-series} argued that some sorts of sensor
networks could decrease the creation of ``noisy data sets" based on the proper
clustering of similar time-series data before they are sent to the data
sink.

Based on the guidelines defined by \cite{db-temporal}, the time properties of
the collected data can be divided into two different dimensions: \textbf{valid
time} and the \textbf{transaction time}. The former depicts the point in time
when the observation took place with respect to the real-world space. An
important aspect of this metadata is that since it captures the time when the
observation occurred, its value cannot be changed during its existence, as well
as placing restrictions on the update operations of this constant. Similarly,
the latter type of is the point in time at which a given fact was transferred
to the persistence layer. Either types of time can be used depending on the
requirements of the sensor network. For instance, \cite{sn-dataware-house}
reports that all the data stored by its sensor network uses the time of
collection, but do not use the transaction time for that purpose.

When both temporal dimensions are applied, the collected data is referred to as
Bitemporal \cite{db-temporal}. Usually, the data type of this metadata is a
regular timestamps, with any supported precision. To clarify, users may be
interested not only when the temperature of 70 degrees happened, but also when
it was collected to the data sink. Finally, \cite{sn-provenance} shows that
the collected data for archival purpose is usually saved in self-labeld
files described by file names, containing complex naming conventions using the 
timestamp as a matter of data index.

\subsection{Data Origin}

Depending on the application of the collected sensor data, the location in
which the data was collected may be needed. For example, \cite{sn-geo-metadata}
shows the geographic location of the collected data is critical to identify
the locations of the events of the sensor device \cite{sn-ex02}. Similarly,
the position of any event captured by a black-box of an aircraft carries 
metadata for the event. Examples of this are the metadata event =
``high-pressure" of 1304, at the altitude of 1,500 feet above the ocean on GPS
coordinates of latitude 13.445 and longitude -23,003. This example clearly
gives necessary identification of where the event of high-pressure occurred.

All in all, Data Provenance enriches the value of the collected data from
sensor devices by describing its identity creation, time of creation and
specific location where the observation occurred. It is important to note,
however, that those types of data do not necessarily need to be on the data
model used, but the specifications of the use of the data determines which
ones are needed. Finally, collected data that are provenance-aware has a direct
use on technologies called Information Fusion \cite{sn-info-fusion}, which aims
at using data from different sensor devices.

\section{Data Load and Management Systems}
\label{sec:data-load}

Depending on the amount of data produced by a sensor network, different
approaches are taken to support the data management depending on
the amount of data collected. For example, \cite{sfbeams2006} is an
environmental sensor network that produces data regarding marine life.
Less than twenty sensor devices produce Megabytes of data on a daily basis,
whose data is processed and stored using flat-files directly in the
operating system's file system with provenance information regarding the time
of collection.

When the amount of data collected grows to other scales, (i.e. Gigabytes) of data
on a daily basis, data becomes more difficult to manage in any type of
database systems. Examples of sensor networks that produce data under this
magnitude are reported in the literature. For instance, 
\cite{sn-dataware-house} describes the creation of a Data Warehouse for the
real-time sensor data management using a cluster of MySQL databases used to
store data from different optical sensors that collect metrics
from physics experiments using eletromagnetic fields. As the author reported,
requirements for data management evolved from storing flat files by a
monolitic system to a robust distributed system using MySQL database.
Moreover, the system targeted from 5 to 10MB of data per second, whose
cumulative data reached 1 TB of data every year.

Data management for large scales of data require different techniques to store
and retrieve data, since it requires more computational power and data
retrieval efficient algorithms. \cite{sn-data-center} describes the creation of
a Sensor Data Center designed to store and analyze amounts of data on the range of
Terabytes to a future Petabytes of data. Its sensor network deals with not only
numerical sensor readings from transportation management, but also radar
images, used for historic purposes and data mining that performs traffic
pattern analysis. In this way, the authors report the use of parallel data
processing techniques in order to provide processing for the collected
data, and load balancing when storing data to minimize the sensor data
processing delays. For this reason, they suggest the use of concurrent control
such as snapshot isolation of the current data into separated replicated data.
Therefore, data access from users is not interrupted by the processing of
the data that is processed.

In summary, this chapter described a subset of problems and solutions used by
the sensor networks community. First, the survey in sensor networks was
reviewd to give a foundation of specific properties of sensor networks. Then,
different surveys in data persistence in sensor networks were discussed,
followed by the data models and query engines. In the end, the most important 
discovery is that data provenance provides the best guidelines for describing 
data. The following chapter will classify the properties found in this chapter
in terms of taxonomies.