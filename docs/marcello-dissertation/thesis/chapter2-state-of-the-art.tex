% main.tex, to be used with thesis.tex
% This contains the main work of your thesis.

%\bibliography{thesis}  % uses the references stored in Chapter1Radar.bib

\chapter{State of the Art}

In general, this chapter reviews surveys and related papers in networks
networks, providing a basic introduction about their characteristics to new 
audience. Similarly, it reviews the characteristics of related works that
discusses and provides data persistence solutions for sensor networks, 
characterizing the different approaches used to described and model the
gathered data from sensor devices.

It is important to note that the scope of this literature review is limited to
the scope of the our case study described in chapter 4 and, therefore, it only
covers a subset of disciplines in sensor networks and data persistence.

\section{Survey in Sensor Networks and Its Infrastructure}

According to \cite{sn-intro02}, sensor networks are specialized types of
network systems comprised of nodes with specific goals such as to produce and
collect data. The former is seen as a sensor device capable of producing data
to the network based on an observation from the environment or simply a raw
data. On the other hand, the latter is described as the final destination of
the raw data in a sensor network and called the network data sink, or simply
data sink, whose responsibility is to receive, categorize and store the
``sensed'' data. In this way, the location of the the data sink is determined by
the purpose of the collected data: data nodes are placed in-network close to
the sensor devices when collected data has a life cycle, being discarded after
being utilized. In contrast, data nodes are placed in in a centralized node
when the collected data is used for historical purposes. 

The following sections describe general properties of sensor networks regarding
infrastructure and components as described in the literature.

\subsection{Properties of Sensor Device Nodes in Sensor Networks}

Each sensor node perform different tasks for the network and, therefore,
influences on the method of deployment used. \cite{sn-intro01} identifies that
the deployment of sensor devices can use arbitrary methods such as being
dropped from an airplane, or manually placed at chosen geographic locations.
Moreover, those authors also mentioned that the deployment can be done as a
one-time event or as an iterative process, where the sensor devices are moved
to different places as needed.

Moving to a differnt point, \cite{sn-intro01} reports that sensor nodes can be
able to move around the place where they were deployed or a stationary sensor
at a given location. In the case of a sensor networks comprised with different
moving sensors, a given sensor node can move occasionally or continuously, as
a result of an accidental side-effect or a desired property of the device. As
a matter of fact, these sensors are automotive or they can be attached to
another moving object. For this reason, \cite{sn-intro01} classifies those
types of mobility as active or passive.

Considering the physical design and capabilities of sensor devices, they range
from microscopically small to bigger devices of a few cubic feet
\cite{sn-intro02}. In addition to their limited size, their computational
resources and power availability are also scarse, when they are located in
areas of limited access to energy. In this way, \cite{sn-intro01} also
distinguishes how sensor devices are usually designed to either store energy on
batteries or use energy scavenged from solar cells. Even though a sensor
network is designed for a purpose, \cite{sn-intro01} also explains that it can
be comprised of different sensor devices playing different roles on what they
observe in the environment, characterizing the heterogeneity the sensor
network by the hardware capabilities and what types of data each of them
produce.

Finally, \cite{sn-intro01} mentions that sensor device nodes can communicate
with each other using different media such as radio, diffuse light or
inductive sound, influencing on the type of protocols they exchange data. They
can have knowledge of their physical location if they are integrated with a
GPS\footnote{Global Positioning System} device, which gives the latitude or
longitude coordinate values. Therefore, their cost of these devices may range
from hundreds to thousands of dollars, depends on a given device's functionality
and capabilities of its design.

\subsection{Properties of the Infrastructure of Sensor Networks}
\label{sec:sn-infrastructure}

Last section discussed the different properties of the sensor devices when
placed in a sensor network. Depending on how they communicate with each other,
\cite{sn-intro02} discusses that sensor devices can be in an organized in a
structured or ad hoc way. For this reason, the infrastructure, of sensor
networks differ in topology such as single-hop, star, tree or  graph. Each of
these topologies may influence on how the produced data from sensor nodes is
collected and routed to their related data sink. While there are only two
levels of node connectivity in sensor networks deployed as single-hop and start
topologies, sensor networks whose topology is either tree or graph use
multi-hop communication among each of their nodes for data communication from
the devices to the data sink.

The property of coverage on a sensor network is defined by
\cite{sn-intro01} as the physical space ``visible" to the sensors devices, which
is also directly related to the purpose of the network, its size in terms of
the number of nodes deployed and the sensor network lifetime. Therefore,
sensors are deployed in sparse or dense manners, according to the
specification of what needs to be covered by the sensor devices. Furthermore,
considering that some types of sensor devices are very likely to regular
hardware failures, \cite{sn-intro02} described sensor network examples with 
redundant nodes that provide better data coverage and levels of coverage.
Finally, the coverage of sensor devices also influence on the connectivity
property of the sensor nodes. In order communicate with redundant nodes, a
physical connection must be available at all times. On the other hand, lower
level of coverage is is related sporadic communication among the sensor nodes.

The sensor network size is related to the degree of coverage mentioned earlier,
and it is measured by the number of devices deployed. \cite{sn-intro01}
reports examples of sensor networks composed by different ranges of deployed
sensor devices such as dozens, hundreds or even thousands of nodes. However,
the life time of a sensor network determines how long these devices are going
to be used, as well as how long the sensor network may exist. For example, an
environmental sensor network \cite{sn-ex01} with the purpose of collecting data
for historic purposes may be have their sensor devices continuously performing
their data collection tasks. On the other hand, however, the lifetime of a
sensor network designed to achieve a goal may not exist after the sensor
devices finish their tasks. Finally, \cite{sn-intro02} argues that the
performance related to the orchestration of sensor devices in sensor networks
is related to the constraints defined for quality-of-service (QoS) such as
real-time data delivery within a thredshold or for robustness such as
maintaining data alive even though a communication link is not available.

All things considered, sensor nodes play an important role during the execution
of a sensor network, whose infrastructure is characterized by how and where the
sensor nodes are deployed and communicate with each other. In this way, next
section covers the approaches to deal with the product of this type of system
network.

\section{Persistence Storage for Sensor Networks}

As traced in the last few sections, sensor networks are comprised of
different types of sensors nodes. However, an important sensor node type is the
data sink because it is the place where the raw data produced by sensor
nodes resides. Data collection and access in sensor networks is directly
related to the characteristics of the sensor network, its routing mechanisms
and where the data sink is placed within the network. In a nutshell,
\cite{sn-storage03} describes collected data as can be described as Low-level
property or an observation such as the level of the ocean, etc.

The type of query mechanism used to directly or indirectly interrogate sensor
devices in a sensor network is directly related to the aforementioned locations
of the collected data. The in-network query processing is used when the
location of the collected data is a local storage \cite{sn-storage01,
sn-storage03} located in the network, strategy typically used to maximize the
sensor's resources, while minimizing energy consumption \cite{sn-storage04}. On
the contrary, the a typical centralized query processing is used to search for
values over the collected data when its location is a centralized data
node \cite{sn-storage02}.

When it comes to the data representation, the relational data model
\cite{relational-model} is one of the most traditional approaches used to
describe the collected data of sensor networks, as it is reported in database
systems such as the TinyDB \cite{sn-db-tinydb}. For instance, \cite{sn-db-newop}
proposes a modified version of the SQL language of TinyDB proposed to solve
SQL Join problems, so that the query processing takes into account the location
of the distributed data. Conversely, \cite{sn-provenance} proposes the use of
Data Provenance as the primary guidelines to describe collected data from
sensor devices, giving the fact that these devices only produce observations as
stream of raw data, and therefore, described to address the problems related to
the origin of the data.

The following sections describe general properties of sensor networks regarding
data persistence, as reviewed in the literature.

\subsection{Data Use and Where Data is Stored}
\label{sec:sn-data-purpose}
\label{sec:sn-storage-locations}

At first glance, the raw data serves as the main outcome of the sensor network.
\cite{sn-provenance} differentiates the use of the collected data as follows as 
\textbf{real-time data stream}, when they are generated data from a sensor
device whenever the data is requested. In this way, the data is still in its raw
format specified by the sensor manufacturer; On the other hand, the authors
describes the data as \textbf{archivable}, since they a have historical
significance and will be processed to be reused by other systems in a latter
time. While the real-time data streams are used and potentially discarded, the
archivable data is stored into secondary memory such as a hard-disk. Under
these circumstances, the collected data needs first to face its journey
starting from its production, transmission from the sensor device, to finally
reach a specified data sink in the network \cite{sn-storage01, sn-storage02,
sn-storage03}.

After producing the ``sensed'' values for the observed parameters from the
environment, the sensor node is responsible to send the raw data to a specific
data node defined by the network infrastructure, as delineated in section
\ref{sec:sn-infrastructure}. In such a way, \cite{sn-storage03}
describes that the location for the collected sensor data provides two
different types of storage devices: the \textbf{external storage} and the
\textbf{local storage}. In the former type, the sensor device sends the
produced data to another sensor node that contains an external storage, whereas
the data only exists in memory or between its reading to its delivery as
real-time data stream in the the latter type. On the whole, the sensor network
infrastructure dictates where the movement performed by the sensor data, as
\cite{sn-storage02} sees a wireless sensor networks as a many-to-one data
collection pattern, when its organization is based on single-hop or star. On
the other hand, data may also flow from one intermediate sensor node to another
until it reaches its final data sink when the data organization is structured
as a graph, tree \cite{sn-storage01, sn-storage03}.

Last, but not least, the use of different external storage devices can be used
to partition the collected data. This practice is called \textbf{Data-Centric
Storage}, and aims at clustering the collected data from different sensor
devices based on a given a property of the data such as location or time of
creation.

\subsection{Query Processing in Sensor Networks}
\label{sec:query-process}

The query processing in sensor networks is a direct or indirect method of
accessing the collected data stored at data nodes, whose location is determined
by different facts described in the previous section. \cite{sn-storage03}
describes an example of an \textbf{in-network query processing} when sensor
nodes, acting as data nodes, provides real-time data stream as a response
to queries requesting the device's collected data. Likewise, if the collected
data is in use for a specific period of time, the data is cached in an external
storage device in a close-by data node. In this way, when a query is issued to
the sensor network with these characteristics, the query processor considers 
all data nodes return its final response to the query. In comparison, a sensor
network whose topology is in form of star, the data sink is usually located in
the center of structure, and thus, receives all the collected data from other
sensor devices in the network. For this reason, it can be seen as a 
\textbf{centralized query processing}. Factoring in that a centralized data
sink receives all the collected data from the peer nodes, this query processing
strategy is the solo responsible for responding to indirect queries about the
produced data from the network. 

Based on the literature review, there are advantages and disadvantages related
each of the query processing approaches described with regards to the
infrastructure of the sensor network. \cite{sn-storage03} argues that the 
primary purpose of the in-network query processing is to minimize the number
of data transmitted from sensor devices to a data sink, as sensor devices are
generally limited by energy use. Considering that sensor devices are also used
as data nodes, \cite{sn-storage04} delineate that this approach can potentially
flood all sensor nodes in the network, and therefore, decrease its performance.
For this reason, \cite{sn-storage01} reports the use of the data-centric
storage by organizing the collected data based on its particular properties
values. In this way, the query processor hits specific data nodes that contains
the types of data in the range of the requested data and, therefore, the author
concludes that this approach decreases the network congestion and maximizes the
capacity on the sensor devices and the performance of the sensor network.

Unlike the in-network, the centralized query processing is related to a regular
database system, located in one of the network nodes. For this reason, this
indirect type of query processing is used when the data purpose of the use of
data is for data archival. However, \cite{sn-storage02} showes that as a
consequence of the many-to-one communication between the centralized data sink
and the other data nodes, this query process does not affect any of the sensor
nodes in the network and, thus, maximizes the sensor device utilization. 
Moreover, this strategy better exposes the collected data to the primary users
raw data, which follows the process of data processing, compression,
aggregation, and so on \cite{sn-db-modeling02}.

Although the approach of centralized query processing maximizes the sensor
nodes capacity and minimize energy consumption, \cite{sn-storage02} explains
that this trade off imposes the creation of the unavoidable development of a
point of traffic concentration or the phenomenon of network congestion. For
this reason, \cite{sn-storage04} describes that different techniques used in 
distributed system such as the use of master-slave data replication or 
database partitioning \cite{db-partitioning-relational} can be used to reduce
or minigate the effects of this funneling effect.

\section{Data Models and Query Engines for Sensor Networks}
\label{sec:data-models}

Different approaches are used to represent the collected data from sensors. The
use of basic data descriptors has been used in research groups which expertise
is not database systems, while others use such technology in pragmatic
approaches. \cite{sn-programming-language} reports different applications using
different types of languages to acces the data modeled in different data
models. \cite{sn-provenance} describes the use of a \textbf{Tabular Data Model}
\cite{tabular-model} using comma-separated values to model and organized the
values related to the properties of observed parameters from a given sensor
device. On the other hand, \cite{sn-db-tinydb} describes the use of the
\textbf{Relational Data Model} \cite{relational-model} is used in the majority
of the projects that involves data persistence due to its popularity in other
different areas of Software Engineering and Distributed Systems. Finally,
\cite{sn-xml-usage01, sn-xml-usage02} are examples of sensor networks
application that makes use of XML \cite{xml} to represent the collected data
on a \textbf{Structured Data Model}.

The following sections describe the application of each of aforenamed data
models in the context of sensor networks.

\subsection{Tabular Data Model and Query}

The most basic computer application to process data is through the use of
spreadsheets, where data is organized into cells \cite{tabular-model}. When
exported to the so-called comma-separated values (CSV) file format, the
resulting file is comprised of lines of ascii characters, separated by a
delimiter. The first line represents a header, with the identification of
columns of specific data, where the following lines contains the values
related to each top column. \cite{sn-provenance} reports that the tabular
data model is commonly used to provide data persistence for sensor networks.
As a matter of fact, the SF-BEAMS environmental sensor networks
\cite{sfbeams2006} uses this approach to distribute the collected data over
the Internet using the OPenDAP \cite{opendap} data format.

This data model can be used in different context, as \cite{sn-provenance}
describes that the versions of the same document can be easily managed and
shared among research colleages using a version-control system such as
Subversion \cite{subversion}.

\subsection{Relational Data Model and its Query Mechanism}

\cite{sn-db-newop} shows the use of the Relational Data Model
\cite{relational-model} can be used in sensor networks to describe the
collected data from sensor devices. As a matter of fact, the project described
by the authors is based on the TinyDB database system \cite{sn-db-tinydb}, a
database system specifically built for data nodes of sensor networks. It uses
the same properties and constraints of regular database system, where the
entities of the sensor network are required to be modeled and represented by
tables.

In order to query the data collected from sensor devices, the use of the 
Structured Query Language (SQL) \cite{sql} is the main language used to extract
the data managed by the system. As mentioned earlier, the default versions of
this language have been modified to give support to the constraints of the
location of the database such as the addition of new SQL operators for TinyDB
\cite{sn-db-newop}.

\subsection{Structured Data Model and its Query Mechanism}

XML is an increasing data format used popularized by the applications developed
for the Internet. \cite{sn-xml-usage01} reports an efficient way to represent
the collected data from sensors using XML, while the authors of
\cite{sn-xml-usage02} describe a sensor network using XML for data model
representation. In general, the collected data is structured into classes of
relating elements and validated by an XML Schema \cite{xml-schema}.

In order to extract data from XML-based databases, the XML XPath language
\cite{xml-xpath} is often used to query XML data. However, with the inception
of hybrid XML-relational database, \cite{sn-xml-usage03} reported the use XML
as the data model and the use of SQL to query the stored data on the IBM's DB2
database system. 

Other examples of the use of the structured data model using XML is the through
the use of middlewares, as the authors of \cite{sn-xml-query-engines} reports.
Such approach is used for data exchange by the implementing of a middleware
capable of query and extract the collected data from sensor devices
\cite{sn-xml-middleware}. A practical example of the use of XML as a middleware
to communicate between nodes is the communication mechanism used by NetBEAMS
\cite{netbeams2009}. See section \ref{sec:dsp-serialization} for details.

\section{How collected data is described?}
\label{sec:sn-data-description}

As described in the previous section, the collected data from sensor networks
can be represented by different data models, requiring the knowledge of the
properties of the data. Not to mention that the low-level or raw data produced
by a given sensor device might not include vital details regarding the
origing and identity of the collected data. For this reason, the collected data
is reported to be described in different ways such as Data Provenance or
Annotations.

\cite{sn-provenance} describes Data Provenance as a technique used to describe
those essential properties of the collected data from sensors. One typical
example of the application of Data Provenance is the so-called blackbox of an
aircraft, which carries the collection of spacio-temporal data structures
collected from a group of sensors placed in different locations of an
aircraft, as shown by \cite{sn-exemple-blackbox}. Last, but not least, the
author describes the collected data as used in a process of data recovery and
analysis in order to identify the probable reason of an aircraft accident, by
using the time, and location of the aircraft, along with the independent
variables from the sensors.

In general, sensor devices produce data and either keep them onto its local
storage or transmit them to data nodes as described earlier in this chapter. It
is vital to understand that the majority of these devices outputs the data in
its raw format, containing only a stream of values with any type of
delimitation between each other. For this reason, the processes of parsing and
description of each of the values located in the data streams regularly
require the use of the device's specification documentation. Similarly, in
order to have access to each of the properties of data collected by a given
property related to time or space, \cite{sn-provenance} also mentions that
Metadata\footnote{Data about data} should be included together with the
collected data.

An example of Data Provenance application is the use of aggregation of data
over time to estimate the changing effects of the weather in a given region.
For this reason, the collected data from sensor devices need meaningful
annotations to better describe the identity of the collected data, as 
\cite{sn-provenance} reports. Interesting questions regarding the
data can be asked such as \textbf{what was collected?}, \textbf{When was the
data collected?} or \textbf{Where was the data collected?}. Similarly, while
Data Provenance provides basic mechanisms to describe sensor networks data in
general, others see the importance of the time space of sensor networks.
\cite{sn-time-series} describes strategies used to cluster distributed sensor
data as time series of data, as well as to decrease the size of data streams
being transmitted in real-time sensor networks when these collected data does
not need to be transferred \cite{sn-data-reduction}.

Another different way to describe sensor devices is through the use of
Annotations. \cite{sn-annotation} shows the use of tags to describe data
collected from a physical area monitoring. The sensor network was
designed to monitor a moving objects, capturing their location while in
movement and their visual appearance using a video camera. In this way,
different instants can be described by adding tags particular video frame, and
therefore, providing an efficient way to search of a particular frame based on
its visual observations. Similarly, this project also uses data provenance to
describe its collected data, since each of the collected data contains the
related GPS coordinates of the observed object.

Finally, data provenance provides practical use of the collected data from
different sensor devices and different sensor networks in the process called
Information Fusison \cite{sn-info-fusion}. An example of the aggregation of
different data from different devices located in different geographic sensor
networks is \cite{sn-geo-metadata}, which is capable of process
spatial-temporal queries of real-time or historical data. The following
sections describe each of the metadata properties as defined by
\cite{sn-provenance}, giving different examples of each of them.

\subsection{What was collected?}

Different sensor devices produce different sets of attributes based on its
capabilities and design. In general, these data sets do not have distinct
names, which leads to difficulties when indices over the properties of the
collected data are required to make them searchable. Specially for the purpose
of data archival, the collected data needs to be identified, giving a meaning
of independence on its own, the same idea behind the unique identification of
entities on the relational data model \cite{relational-model}. Equally,
\cite{sn-provenance} describes that traditional sensor networks uses the
collected data for analysis long after the attributes have been collected,
usually before it is categorized and indexed. Annotations and tags are also
names given by \cite{sn-provenance} as means to describe collected sensor data.

In a nutshell, the collected data is described by the name of the observation
property. For instance, \textbf{temperature} = 73.45 expresses the value of the
observation of the temperature. In other words, the string ``temperature'' is
the property describing the raw collected data 73.45. On the other hand,
however, the example \textbf{temperature-scale} = ``Fahrenheit": describes
the metadata regarding the temperature.

\subsection{When was the data collected?}

In order to use the collected data for the purpose of historical analysis, the
use of a point in time is the most prominent feature of data. Also referred to
as spatial-temporal data, it is considered a pivotal metadata used in different
types of sensor networks such as an aircraft the blackbox
\cite{sn-exemple-blackbox}, environmental sensor network \cite{sfbeams2006},
and data communications management systems \cite{sn-dataware-house}. In this
way, Other projects classify data in time-series \cite{sn-time-series-example},
where data is related to all the observations a given sensor device performed
over a period of time. The apparent implications is that all the produced data
is necessary, but \cite{sn-time-series} argued that some sorts of sensor
networks could decrease the creation of ``noisy data sets" based on the proper
clustering of similar time-series data before before they are sent to the data
sink.

Based on the guidelines defined by \cite{db-temporal}, the time properties of
the collected data can be divided into two different dimensions: \textbf{valid
time} and the \textbf{transaction time}. The former depicts the point in time
when the observation took place with respect to the real-world space. An
important aspect of this metadata is that since it captures the time when the
observation occurred, its value cannot be changed during its existence, as well
as placing restrictions on the update operations of this constant. Similarly,
the latter type of is the point in time at which a given fact was transferred
to the persistence layer. Either types of time can be used depending on the
requirements of the sensor network. For instance, \cite{sn-dataware-house}
reports that all the data stored by its sensor network uses the time of
collection, but do not use the transaction time for that purpose.

When both temporal dimensions are applied, the collected data is referred to as
Bitemporal \cite{db-temporal}. Usually, the data type of this metadata is a
regular timestamps, with any supported precision. To clarify, users may be
interested not only when the temperature of 70 degrees happened, but also when
it was collected to the data sink. Finally, \cite{sn-provenance} shows that
the collected data for archival purpose is usually saved in self-described
files described by file names, containing complex naming conventions using the 
timestamp as a matter of data index.

\subsection{From where was the data collected?}

Depending on the application of the collected sensor data, the location in
which the data was collected may be needed. For example, \cite{sn-geo-metadata}
shows the geographic location of the collected data is critical to identify
the locations of the events of the sensor device \cite{sn-ex02}. Similarly,
the position of any event captured by a black-box of an aircraft carries 
metadata for the event. Examples of this are the metadata event =
``high-pressure" of 1304, at the altitude of 1,500 feet above the ocean on GPS
coordinates of latitude 13.445 and longitude -23,003. This example clearly
gives necessary identification of where the event of high-pressure occurred.

All in all, Data Provenance enriches the value of the collected data from
sensor devices by describing its identity creation, time of creation and
specific location where the observation occurred. It is important to note,
however, that those type of data do not necessarily need to be on the data
model used, but the specifications of the use of the data determines which
ones are needed. Finally, collected data that are provenance-aware has a direct
use on technologies called Information Fusion \cite{sn-info-fusion}, which aims
at using data from different sensor devices.

\section{Data Load and Management Systems}
\label{sec:data-load}

Depending on the amount of data produced by a sensor network, different
approaches are reported to be taken to support the data management depending on
the amount of data collected. For example, \cite{sfbeams2006} is a an
environmental sensor network that produces data regarding the marine life.
Less than twenty sensor devices produce Megabytes of data in a daily bases,
whose data is processed and stored in using flat-files directly in the
operating system's file system with provenance information regarding the time
of collection.

When the amount of data collected grows to other scales, say, Gigabyes of data
in a daily basis, data management tends to become more difficult in any type of
database systems. Examples of sensor networks that produce data under this
magnitude of data are reported in the literature. For instance, 
\cite{sn-dataware-house} describes the creation of a Data Warehouse for the
real-time sensor data management using a cluster of MySQL databases used to
store data from different sensors such as optical sensors that collect metrics
from physics experiments using eletromagnetic fields. As the author reported,
requirements for data management evolved from storing flat files by a
monolitic system to a robust distributed system using MySQL database.
Moreover, the the system targed from 5 to 10MB of data per second, whose
culmulative data reached 1 TB of data every year.

Data management for large scales of data require different techniques to store
and retrieve data, since it requires more computational power and data
retrieval efficient algorithms. \cite{sn-data-center} describes the creation of
a Sensor Data Center designed to store and analyze amounts of data on the range of
Terabytes to a future Petabyes of data. Its sensor network deals with not only
numerical sensor readings from transportation managemenet, but also radar
images, used for historic purposes and data mining that performs traffic
pattern analysis. In this way, the authors report the use of parallel data
processing techniques in order to provide processing for the collected
data, and load-balancing when storing data to minimize the sensor data
processing delays. For this reason, they suggest the use of concurrent control
such as snapshot isolation of the current data into separated replicated data.
Therefore, data access from users are not interrupted by the processing of
the data that is processed.

All in all, this chapter described a subset of problems and solutions used by
the sensor networks community. First, the survey in sensor networks was
described to give a foundation of specific properties of sensor networks. Then,
different surveys in data persistence in sensor networks were reviewed,
followed by the data models and query engines. Last, but not least, different
project described what is important to describe the collected to finally verify
different properties of data load and management. The following chapter will
clssify the properties found in this chapter in terms of taxonomies.
