% main.tex, to be used with thesis.tex
% This contains the main work of your thesis.

%\bibliography{thesis}  % uses the references stored in Chapter1Radar.bib

\chapter{DSP Data Persistence: Component Implementation}

This section shows the implementation of the DSP Data Persistence, which
follows the specifications of the DSP Components described on section 2.4,
along with the setup of the mongoDB, the chosen database system that handles
document-oriented instances. The implementation of the DSP Data Persistence
component was developed using the DSP Subversion repository at
http://code.google.com/p/netbeams.

\section{DSP Deployment on OSGi Framework}

    * Description of the OSGi MANIFEST.MF
    * Description of the component on the config.xml
    * Adding a matching rule on the matcher\underline{ }config.xml
    * Adding a bootstrap message the component and database configuration


    * Adding a new entry into the deployment configuration artifact config.xml;
    * Adding a new entry into the matcher configuration artifact matcher\underline{ }config.xml;
    * Adding an optional configuration message that sets up the new DSP Component;
    * Adding new Java drivers that communicates with the Database system;
    * Installing and configuring the proposed database system.

\subsection{Starting the DSP Platform}

    * command-line (Knopflerfish Framework)
    * user interface (Knopflerfish Framework)

\subsection{Stopping the the DSP Platform}

    * command-line (Knopflerfish Framework)
    * user interface (Knopflerfish Framework)

\subsection{Execution Logs}

\section{mongoDB Deployment}

Given that Document-Oriented Model makes a good candidate persist sensors'
properties and the recently enumerated list technologies in the previous
article DSPDataPersistence, the open-source project called mongoDB was chosen
for the evaluation on our case study, the netBEAMS DSP Platform.

mongoDB supports storage based on collections of data, stored using BSON, a
binary representation the JSON data representation format, including dynamic
queries and indexing support. As it's stated in their web site, mongoDB
"bridges the gap between key/value stores (which are fast and highly scalable)
and traditional RDBMS systems (which are deep in functionality)".

\begin{itemize}
  \item mongoDB implements a document-oriented structure, which is similar to
  KVP;
  \item mongoDB is written in C++, and therefore, can is available in any major
  platform, as well as offers a broad range of API drivers written in different
  languages such as Java, Python, Perl and Ruby; 
  \item mongoDB is open-source, with good community support and availability
  through mailing lists, freenode IRC channel, and commercial support through
  10gen company; 
  \item mongoDB has support to distributed systems properties such as
  Master-Slave replication, and features like Database Shards with
  auto-sharding based on shard keys.
\end{itemize} 

nstallation of mongoDB
- Explanation about mongoDB processes: mongod, mongos, mongo
- Specification of the shards, shard key, etc

    * The database instance is called "netbeams";
    * The database "netbeams" may contain different collections, categorized by
    the Sensor Content Type, that is, depending on how the DSP Component was
    described;  

\subsection{Document-Oriented Data Model}

An example of the data is as folows (using the JSON syntax). Listing
\ref{file:mongodb-ysi-data-format} follows the strategy described on Section X,
where the entity representation is denormalized, and each of the properties of
the sensor is repeated in each instance of the data. The fact and transaction
times are expressed in milliseconds. The data element contains the complete
structure of the originating sensor such as the IP address. The collection of
instances of this document represents the sensor data type.

\subsection{Starting the Database}

\subsection{Stopping the Database}

\subsection{Execution Logs}

- Java API implementation
- Python API implementation

\section{Data Access Using Database Shell}

The mongoDB client can be started by using the following command. Make sure you
have started the mongoDB server before executing the mongoDB client.


mongo netbeams | tee output\underline{ }number\underline{ }date.log


Here, the iterative mongo client shell offers users to verify and navigate on a
given database and its collections. This first section shows the connection of
the mongo client to the database netbeams. It also highlights the query for the
collections available. During the experiment, the SondeDataContainer?
collection was created as related to the type from the DSP Messages for the YSI
Sonde.

The shell references to the mongoDB system can be found at
http://www.mongodb.org/display/DOCS/dbshell+Reference 

