% main.tex, to be used with thesis.tex
% This contains the main work of your thesis.

%\bibliography{thesis}  % uses the references stored in Chapter1Radar.bib

\chapter{Experimental Results: Correct Behavior and Performance}

In order to evaluate the mongoDB, the different taxonomies identified are
verified from the logs and experiment setup. The implementation of the Data
Persistence Component and the mongoDB infrastructure are described in section
\ref{sec:dsp-data-persistence-implementation}.

The purpose of Sensor Data is used for Data Archival, since the data from the
sensor data categorized by the date and time it was collected. The location
of the collected data from the sensor devices are stored using the External
Storage, or using the Data-Centric Strategy. In this way, one of the advantages
of the denormalized data model provided by mongoDB is that the data can be
distributed using Horizontal Partitioning.

\section{Experiment Setup}

Experiments were conducted to evaluate mongoDB against the taxonomies defined in
chapter 3 by creating a general shell script
\ref{file:experiment-setup-executor} that sets up the environment by
initializing the mongoDB server and indirectly inserts the a desired number of
randomly generated YSI Data types \ref{file:random-ysi-data-generator}
implemented in Java as part of the NetBEAMS data collection mechanism as shown
in section \ref{sec:dsp-payload-implementation}. First, the definition of a 
document will be conducted according to the definition of the Data Provenance
definition of chapter 3. Then, the infrastructure of the database will be
evaluated using different scenarios.

\subsection{Key-Value definition}

In document-oriented databases, there is no definition of table. In constrast,
the entities are persisted using any data arrangment, in any order, attributes
and the associated values. The definition of the document for mongoDB takes
into account the properties defined by the data provenance study defined by
\cite{sn-provenance}. The following list are the definitions of the values of
the defined keys and the provenance-aware document, as an example described by
Listing \ref{file:mongodb-ysi-data-format}:

\begin{itemize}
  \item \textbf{What: Data Identity}: the identification of the data captured
  from the sensor, as well as the DSP Platform message. The keys are comprised
  by the attributes ``\underline{ }id'' as an unique identifier provided by
  mongodb, the ``message\underline{ }id'';
  \item \textbf{When: Time Dimension}: the time dimensions give information
  regarding when the data was collected and transferred to the network sink.
  They are the ``time.valid'', ``time.transaction'';
  \item \textbf{Where: Data Location}: the location where 
 the data originated. The ``sensor.ip\underline{ }address'',
 ``sensor.location.latitude'', ``sensor.location.longitude'' give information
 regarding the .
\end{itemize}

\subsection{Workload}

The workload selected for the experiments reflacts the current use of the data
at the SF-BEAMS infrastructure. A number of data worth a year of collected data
from one single YSI device is randomly generated and inserted at the database.
Then, five additional run will be performed and investigate different search
scenarios for performance.

In order to verify the system capacity to scale, the data will be devided into
different shards as part of the support to distribute system, that is, in a
Data-Centric way.

\begin{itemize}
  \item First Round: 483,840 items worth one year of produced YSI data at the
  rate of 1 observation per minute;
  \item 483,840 * 5, or 2,419,200, representing 5 different YSI components
  producing data every minute;
\end{itemize}

\subsection{Single Server Environment}

The definition for the single server follows a regular specification of the
mongoDB database system. All the collected data from sensors are stored in a
given Single Server, which characterizes an External Storage device.

\subsection{Data-Centric Environment}

The definition of a Data-Centric deployment follows the specification of the
mongoDB shard servers, where in contrast to the single server, all the
collected data is spread out to different mongoDB clients that manage
partitions of servers on a cluster. 

As described in the taxonomy reviews, the definition of the sharded system must
include the definition of a shard key, which is responsible for partitioning
the instances. In this way, the shard key is defined as the value of the
attribute ``sensor.id\underline{ }address''.

\subsection{Scenarios}

In order to verify the feasibility of the system, the use of different search
scenarios are used:

\begin{itemize}
  \item A) Find data by specific date ranges on either the valid or the
  transaction time;
  \item B) Find data by specific IP address, as it identifies a given sensor
  device and any data produced;
  \item C) Find data by specific values of the observed attributes of the sensor
  device such as temperature, salinity, etc.
\end{itemize}

\section{Measurements Results}

The result of the insertion of data in different scenarios are shown in the
different logs captured during the experiments. First, the avarage insertion
times gives the capacity of the system with and without the definition of
indexes over the data. Then, the numbers regarding the scenarios described
earlier will be shown:

\subsection{Single Server Environment}

The insersion time avarages are regarding the different approaches to model a
document in the database system:

\begin{itemize}
  \item No Index : Insert Avarage = xyz 
  \item Indexed : Insert Avarage = xyz
\end{itemize}

On the other hand, the avarages for each of the scenarios are devided by the
amount of data produced by one single device or five different devices.

For one single device, the values can be summarized as follows:
\begin{itemize}
  \item Not Index : A) = xyz
  \item Indexed : A) = xzy
  \item Not Index : B) = xyz
  \item Indexed : B) = xzy
  \item Not Index : C) = xyz
  \item Indexed : C) = xzy
\end{itemize}

For data worth five different devices, the values can be summarized as follows:
\begin{itemize}
  \item Not Index : A) = xyz
  \item Indexed : A) = xzy
  \item Not Index : B) = xyz
  \item Indexed : B) = xzy
  \item Not Index : C) = xyz
  \item Indexed : C) = xzy
\end{itemize}

\subsection{Data-Centric Cluster Environment}

The insertions in a sharded environment are added the operation of finding
which ``bucket'' the data must be placed. mongoDB gives support for automatic
sharding after the definition of a key based on range of the values. The
avarages of the insertions can be shown as follows:

\begin{itemize}
  \item NO Index : Insert Avarage = xyz
  \item Indexed : Insert Avarage = xyz
\end{itemize}

For one single device, the values can be summarized as follows:

\begin{itemize}
  \item Not Index : A) = xyz
  \item Indexed : A) = xzy
  \item Not Index : B) = xyz
  \item Indexed : B) = xzy
  \item Not Index : C) = xyz
  \item Indexed : C) = xzy
\end{itemize}

For data worth five different devices, the values can be summarized as follows:

\begin{itemize}
  \item Not Index : A) = xyz
  \item Indexed : A) = xzy
  \item Not Index : B) = xyz
  \item Indexed : B) = xzy
  \item Not Index : C) = xyz
  \item Indexed : C) = xzy
\end{itemize}

\section{Discussion}

mongoDB is that the data can be distributed using Horizontal Partitioning,
which mitigates problems related to referencial integrity.

\begin{itemize}
  \item Strengths as compared to literature
  \item Weaknesses as compared to literature
\end{itemize}

Data-Centric Storage and Database Shards = Horizontal scalability for data
based on specific keys = Scalability facing High Volume of Collected Data

Weakness, Technological, and skills-based.