% main.tex, to be used with thesis.tex
% This contains the main work of your thesis.

%\bibliography{thesis}  % uses the references stored in Chapter1Radar.bib

\chapter{Conclusions and Future Works}

This work presented a collection of taxonomies related to data persistence for
sensor networks using different approaches ranging from data models and server
infrastructure organization. The main result of this work was that persistence
for collected data from sensors are better maintained using schema-less data
models, since it provides offers an approach to deal with unanticipated data
types from the sensor devices.

There are many important features covered in this work, including the
importance of the sensor networks rely in how users have access to the
collected data. In order to provide access to the collected data, one must
first assess the different types of infrastructural characteristics of the
studied sensor network. In general, the selection of a technology that
provides such persistence capabilities can be part of a process of analysis of
taxonomies proposed by this work. However, this work proposes a data model not
yet used in the sensor network community and, thus, represents an important
contribution for the area.

Taking into account the software architecture provided by NetBEAMS, as well as
the requirements for data collection from the RTC research group, this work
shows a novice way to persist data collected from sensor devices by using a
schema-less data model, whose query capabilities are based upon programming
languages, in contrast to the use of the Structured Query Language, SQL. The
experiment results showed that the implementation of a persistence layer with
mongoDB can be used to persist data at its maximum rate of one sample per
minute. However, performance can be improved by using techniques such as
database sharding, since it offers a direct implementation capability to
Data-Centric Storage approach. This approach decreases the amount of time to
search data since it distributes the data instances across different cluster
nodes.

There are many different directions for future works. First, the Data-Centric
implementation performance may be extremely improved by using the technique
called MapReduce \cite{map-reduce}. This approach allows mongoDB to search
different machines in parallel whenever the data requested is located in
different database shard hosts. In addition to that, the use techniques to
cluster collected data before they are inserted into the database may be
important as shown in \cite{sn-time-series}. One of the key issues in the
implemented solution is that any measurement of data is considered to be
persisted without further analysis.

All in all, the DSP Data Persistence component implemented for the NetBEAMS
solution to SF-BEAMS can be used in any network server. The solution for data
persistence is a novice way to save data into the database, provided by
mongoDB's ability to cope with the uncertainty of any properties collected from
any sensor specification. Furthermore, the solution also provides a good
support to the provenance-related data.