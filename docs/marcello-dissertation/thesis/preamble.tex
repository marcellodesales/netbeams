% preamble.tex, to be used with thesis.tex
% This contains the TeX definitions for layout, style, etc., as well as the first few pages of your thesis: title page, copyright page, approval page, abstract, acknowledgments, tables of contents, tables, and figures.
% The layout commands should give the correct margins according to the graduate division's guidelines

%%%%% TeX class and packages

\documentclass[12pt,oneside]{sfsuthesis}
\usepackage{amsthm,amsmath,amssymb,amsfonts,latexsym,graphicx,enumerate,setspace,verbatim,makeidx,subfigure, multirow, url, soul}
\usepackage[subfigure]{tocloft}
\usepackage{color}                       % For creating colored text and background
%\usepackage{hyperref}                 % For creating hyperlinks in cross
%references
% other possibly useful packages: textcomp,mathrsfs,amscd,epsfig,euscript,cancel

%%%%% Layout
% These numbers might depend on your printer. Check the margins and compare them to the Graduate Division's
% guidelines. If there's something off, try playing with the numbers...
%
% For chapters:
%     Must have a minimum of 1.5in margin on left and 1in on all other sides.  Where there are page numbers
%     (whether on top or bottom), must have one additional inch between the page number and the text, for a
%     total of 2in between the edge of the paper and the text.
% For frontmatter pages:
%     The same margin numbers generally work, except for the Title Page, so you will notice that we use
%     some numbers for \textheight and \footskip right here, and then change them below, right after
%     generating the Title Page.

\hoffset=.5in 
\oddsidemargin=0in
\evensidemargin=0in
\topmargin=0.1in 
\headheight=0in
\headsep=1in
\footskip=.8in
\textwidth=5.9in
\textheight=7in

\setcounter{secnumdepth}{3}
\setcounter{tocdepth}{3}

\pagestyle{plain}

\doublespacing

%%%%% Style of theorems, definitions, examples, equations, etc.

\theoremstyle{plain} % Heading is bold, text italic.
\newtheorem{theorem}{Theorem}[chapter]
\newtheorem{lemma}[theorem]{Lemma}
\newtheorem{proposition}[theorem]{Proposition}
\newtheorem{corollary}[theorem]{Corollary}
\newtheorem{conjecture}{Conjecture}[chapter]

\theoremstyle{definition}  % Heading is bold, text is roman
\newtheorem{definition}{Definition}[chapter]
\newtheorem{example}{Example}[chapter]

\theoremstyle{remark}  % Heading is italic, text is roman
\newtheorem*{remark}{Remark}
\newtheorem*{note}{Note}
\newtheorem{claim}{Claim}[chapter]

%%%%% Appendix style

\renewcommand\appendix[1]{
\chapter*{#1}
\addcontentsline{toc}{chapter}{#1}
}

%%%%% Source-Code listings

\usepackage{courier}
\usepackage{color}
\usepackage{xcolor}
\usepackage{listings}

\lstset{commentstyle=\tiny,captionpos=t,tabsize=2,frame=lines,keywordstyle=\color{blue}\tiny,commentstyle=\color{gray}\tiny,stringstyle=\color{red}\tiny,numbers=left,numberstyle=\tiny,breaklines=true,showstringspaces=false,basicstyle=\tiny,emph={label}}

%\definecolor{darkgreen}{named}{green}
%\definecolor{darkblue}{named}{blue}
%\definecolor{darkred}{named}{red}
%\definecolor{grau}{named}{gray}
%\let\Righttorque\relax
%\lstset{
%commentstyle=\itshape\color{darkgreen},
%keywordstyle=\bfseries\color{darkblue},
%stringstyle=\color{darkred},
%extendedchars=true,
%basicstyle=\scriptsize\ttfamily,
%basicstyle=\tiny\ttfamily,
%tabsize=2,
% keywordstyle=\textbf,
%commentstyle=\color{grau},
% stringstyle=\textit,
%numbers=left,
%numberstyle=\tiny,
% für schönen Zeilenumbruch
%breakautoindent = true,
%breakindent = 2em,
%breaklines = true,
%postbreak = ,
%prebreak = \raisebox{-.8ex}[0ex][0ex]{\ensuremath{\lrcorner}},
%prebreak = \raisebox{-.8ex}[0ex][0ex]{\Righttorque},
%showspaces=false, % Keine Leerzeichensymbole
%showtabs=false, % Keine Tabsymbole
%showstringspaces=false,% Leerzeichen in Strings
%}

\usepackage{caption}
\DeclareCaptionFont{white}{\color{white}}
\DeclareCaptionFormat{listing}{\colorbox{gray}{\parbox{\textwidth}{#1#2#3}}}
\captionsetup[lstlisting]{format=listing,labelfont=white,textfont=white}

%% Use fancy chapter headers, with Jos Dingjan's modifications,
%% plus my own tweaks. This style is not part of teTeX,
%% so we are using a local (and renamed) copy.
\usepackage[Lenny]{fncychap}

%Hifens
\usepackage[latin1]{inputenc}
\usepackage[USenglish]{babel}

%Harvard citation style
\usepackage{natbib}
%enhanced items
% http://dante.ctan.org/tex-archive/macros/latex/contrib/enumitem/enumitem.pdf
\usepackage{enumitem}

\begin{document}

\pagenumbering{roman}
\thispagestyle{empty}

\[ \]
\vspace{-1.8in}

\begin{center}
{\mytitle}

\vspace{1.4in}

\singlespace{A thesis presented to the faculty of\\
San Francisco State University\\
In partial fulfillment of\\
The requirements for\\ The degree}

\vspace{.5in}

\singlespace{Master of Science\\ In\\ Computer Science}

\vspace*{\fill}

{by \\[12pt] 
\myname \\[12pt]
San Francisco, California\\[12pt]
\thismonth
\thisyear}
\end{center}

\newpage
\thispagestyle{empty}

$\mbox{}$
\vspace{3in}
\begin{center}
\singlespace{
    Copyright \copyright{} \thisyear by \myname

    \bigskip

    All rights reserved. No part of the material protected by this
    copyright notice may be reproduced or utilized in any form or by any
    means, electronic or mechanical, including photocopying, recording or
    by any information storage and retrieval system, without the prior
    permission of the author.
}
\end{center}

\newpage
\thispagestyle{empty}
\[ \]
\vspace{-1.8in}
\begin{center}
{CERTIFICATION OF APPROVAL}
\end{center}
\vspace{.5in}
\begin{quote}
I certify that I have read {\it \mytitle} by \myname, and that in my opinion
this work meets the criteria for approving a thesis submitted in partial
fulfillment of the requirements for the degree: Master of Science in Computer
Science at San Francisco State University.
\end{quote}

\vspace{1.5in}

\hspace*{\fill}\parbox{3.5in}{
\singlespace{

\hrule{\hspace{3.5in}} \\ 
Prof. Arno Puder, Ph.D.\\
Professor of Computer Science\\
\vspace{1in}
\hrule{\hspace{3.5in}} \\
Prof. Marguerite C. Murphy, Ph.D.\\
Professor of Computer Science

}
}

\newpage
\thispagestyle{empty}
\[ \]
\vspace{-1.8in}
\begin{center}
{\mytitle} \\

\vspace{.5in}

\singlespace{
\myname \\
San Francisco, California \\  
\thisyear \\
}
\end{center}

\vspace{.5in}

%\doublespacing{\noindent
\onehalfspacing{\noindent      
Sensor Networks are becoming more and more important to different science
communities bacause of what they produce: the raw data for different studies.
In order to make use the collected data, researchers may have to dissect the
characteristics of the sensor network in question, regarding different
properties such as the purpose and location of the observed data, as well as
how the data is described. For this reason, this work proposes a set of data
persistence taxonomies based on the state of the art, which can be used to
classify the properties of the produced raw data. In order to evaluate the
proposed taxonomies, this work designs and implements a data persistence
component for NetBEAMS, our case study of modular sensor network infrastructure
developed to improve the operation of the SF-BEAMS environmental sensor
network. In this way, based on empirical analysis regarding proposed
taxonomies for the case study, this work selected mongoDB, a schema-less
document-oriented database instead of the traditional use of relational
databases. As a result based on the experiments conducted, this work suggests
that use of Key-Value Pair datasents in orde tor provide External of
Data-Centric approach. Therefore, the use of a solution of the help decreasing
the complexity related to the data persistence layer for sensor networks with
external or data-centric storage mechanism, and whose purpose is data
archival. Finally, this work proposes different future works such a
data-centric approach using Databa se Shards to store te collected data and Map
Reduce to execute parallel queries over the shards.} \vspace*{\fill}
 
\hspace*{\fill}

\noindent
I certify that the Abstract is a correct representation of the content of this thesis.

\vspace{.5in} 

\hrule{\hspace{3.75in}} \\[-10pt]
Chair, Thesis Committee 
\hspace{2.5in}
Date
\doublespacing
\newpage
\[ \]
\vspace{-1.8in}
\begin{center}{ACKNOWLEDGMENTS}\end{center}

\vspace{.5in}
\begin{quote}
\noindent
Acknowledgments
\end{quote}

\renewcommand{\contentsname}{\vspace{-1.7in} \begin{center} \normalsize \rm TABLE OF CONTENTS \end{center}}
\renewcommand{\listfigurename}{\vspace{-1.7in} \begin{center} \normalsize \rm LIST OF FIGURES \end{center}}
\renewcommand{\listtablename}{\vspace{-1.7in} \begin{center} \normalsize \rm LIST OF TABLES \end{center}}
\renewcommand{\cftchapfont}{\normalfont}
\renewcommand{\cftchappagefont}{\normalfont}
\renewcommand{\cftchapleader}{\cftdotfill{\cftdotsep}} % formatting commands for table of contents
\renewcommand{\cftsecfont}{\normalfont}
\renewcommand{\cftsecpagefont}{\normalfont}
\renewcommand{\cftsecleader}{\cftdotfill{\cftdotsep}}

\newpage \tableofcontents 
\newpage \listoftables % comment out if you don't use tables
\newpage \listoffigures % comment out if you don't use figures

\newpage
\pagestyle{myheadings}
\pagenumbering{arabic} 
\setcounter{page}{1}
