% preamble.tex, to be used with thesis.tex
% This contains the TeX definitions for layout, style, etc., as well as the first few pages of your thesis: title page, copyright page, approval page, abstract, acknowledgments, tables of contents, tables, and figures.
% The layout commands should give the correct margins according to the graduate division's guidelines

%%%%% TeX class and packages

\documentclass[12pt,oneside]{sfsuthesis}
\usepackage{amsthm,amsmath,amssymb,amsfonts,latexsym,graphicx,enumerate,setspace,verbatim,makeidx,subfigure, multirow, url, soul}
\usepackage[subfigure]{tocloft}
\usepackage{color}                       % For creating colored text and background
%\usepackage{hyperref}                 % For creating hyperlinks in cross
%references
% other possibly useful packages: textcomp,mathrsfs,amscd,epsfig,euscript,cancel

%%%%% Layout
% These numbers might depend on your printer. Check the margins and compare them to the Graduate Division's
% guidelines. If there's something off, try playing with the numbers...
%
% For chapters:
%     Must have a minimum of 1.5in margin on left and 1in on all other sides.  Where there are page numbers
%     (whether on top or bottom), must have one additional inch between the page number and the text, for a
%     total of 2in between the edge of the paper and the text.
% For frontmatter pages:
%     The same margin numbers generally work, except for the Title Page, so you will notice that we use
%     some numbers for \textheight and \footskip right here, and then change them below, right after
%     generating the Title Page.

\pdfpagewidth 8.5in
\pdfpageheight 11in 
\setlength\topmargin{0in}
\setlength\headheight{0in}
\setlength\headsep{0in}
\setlength\textheight{7.7in}
\setlength\textwidth{6.5in}
\setlength\oddsidemargin{0in}
\setlength\evensidemargin{0in}
\setlength\headheight{77pt}
\setlength\headsep{0.25in}

%\hoffset=.5in 
%\oddsidemargin=0in
%\evensidemargin=0in
%\topmargin=0.0in 
%\headheight=0in
%\headsep=1in
%\footskip=.8in
%\textwidth=6.5in
%\textheight=7in

\setcounter{secnumdepth}{3}
\setcounter{tocdepth}{3}

\pagestyle{plain}

\doublespacing

%%%%% Style of theorems, definitions, examples, equations, etc.

\theoremstyle{plain} % Heading is bold, text italic.
\newtheorem{theorem}{Theorem}[chapter]
\newtheorem{lemma}[theorem]{Lemma}
\newtheorem{proposition}[theorem]{Proposition}
\newtheorem{corollary}[theorem]{Corollary}
\newtheorem{conjecture}{Conjecture}[chapter]

\theoremstyle{definition}  % Heading is bold, text is roman
\newtheorem{definition}{Definition}[chapter]
\newtheorem{example}{Example}[chapter]

\theoremstyle{remark}  % Heading is italic, text is roman
\newtheorem*{remark}{Remark}
\newtheorem*{note}{Note}
\newtheorem{claim}{Claim}[chapter]

%%%%% Appendix style

\renewcommand\appendix[1]{
\chapter*{#1}
\addcontentsline{toc}{chapter}{#1}
}

%%%%% Source-Code listings

\usepackage{courier}
\usepackage{color}
\usepackage{xcolor}
\usepackage{listings}

\lstset{commentstyle=\tiny,captionpos=t,tabsize=2,frame=lines,keywordstyle=\color{blue}\tiny,commentstyle=\color{gray}\tiny,stringstyle=\color{red}\tiny,numbers=left,numberstyle=\tiny,breaklines=true,showstringspaces=false,basicstyle=\tiny,emph={label}}

%\definecolor{darkgreen}{named}{green}
%\definecolor{darkblue}{named}{blue}
%\definecolor{darkred}{named}{red}
%\definecolor{grau}{named}{gray}
%\let\Righttorque\relax
%\lstset{
%commentstyle=\itshape\color{darkgreen},
%keywordstyle=\bfseries\color{darkblue},
%stringstyle=\color{darkred},
%extendedchars=true,
%basicstyle=\scriptsize\ttfamily,
%basicstyle=\tiny\ttfamily,
%tabsize=2,
% keywordstyle=\textbf,
%commentstyle=\color{grau},
% stringstyle=\textit,
%numbers=left,
%numberstyle=\tiny,
% für schönen Zeilenumbruch
%breakautoindent = true,
%breakindent = 2em,
%breaklines = true,
%postbreak = ,
%prebreak = \raisebox{-.8ex}[0ex][0ex]{\ensuremath{\lrcorner}},
%prebreak = \raisebox{-.8ex}[0ex][0ex]{\Righttorque},
%showspaces=false, % Keine Leerzeichensymbole
%showtabs=false, % Keine Tabsymbole
%showstringspaces=false,% Leerzeichen in Strings
%}

\usepackage{caption}
\DeclareCaptionFont{white}{\color{white}}
\DeclareCaptionFormat{listing}{\colorbox{gray}{\parbox{\textwidth}{#1#2#3}}}
\captionsetup[lstlisting]{format=listing,labelfont=white,textfont=white}

%% Use fancy chapter headers, with Jos Dingjan's modifications,
%% plus my own tweaks. This style is not part of teTeX,
%% so we are using a local (and renamed) copy.
\usepackage[Lenny]{fncychap}

%Hifens
\usepackage[latin1]{inputenc}
\usepackage[USenglish]{babel}

%Harvard citation style
\usepackage{natbib}
%enhanced items
% http://dante.ctan.org/tex-archive/macros/latex/contrib/enumitem/enumitem.pdf
\usepackage{enumitem}

\begin{document}

\pagenumbering{roman}
\thispagestyle{empty}

\[ \]
\vspace{-1.8in}

\begin{center}
{\mytitle}

\vspace{1.4in}

\singlespace{A report presented to the faculty of\\
San Francisco State University\\
In partial fulfillment of\\
The requirements for\\ The degree}

\vspace{.5in}

\singlespace{Master of Science\\ In\\ Computer Science}

\vspace*{\fill}

{by \\[12pt] 
\myname \\[12pt]
San Francisco, California\\[12pt]
\thismonth
\thisyear}
\end{center}

\newpage
\thispagestyle{empty}

$\mbox{}$
\vspace{3in}
\begin{center}
\singlespace{
    Copyright \copyright{} \thisyear by \myname
%    \bigskip
%    All rights reserved. No part of the material protected by this
%    copyright notice may be reproduced or utilized in any form or by any
%    means, electronic or mechanical, including photocopying, recording or
%    by any information storage and retrieval system, without the prior
%    permission of the author.
}
\end{center}

\newpage
\thispagestyle{empty}
\[ \]
\vspace{-1.8in}
\begin{center}
{CERTIFICATION OF APPROVAL}
\end{center}
\vspace{.5in}
\begin{quote}
I certify that I have read {\it \mytitle} by \myname, and that in my opinion
this work meets the criteria for approving a culminating experience submitted in
partial fulfillment of the requirements for the degree: Master of Science in Computer
Science at San Francisco State University.
\end{quote}

\vspace{1.5in}

\hspace*{\fill}\parbox{3.5in}{
\singlespace{

\hrule{\hspace{3.5in}} \\ 
Prof. Arno Puder, Ph.D.\\
Professor of Computer Science\\
\vspace{1in}
\hrule{\hspace{3.5in}} \\
Prof. Marguerite C. Murphy, Ph.D.\\
Professor of Computer Science

}
}

\newpage
\thispagestyle{empty}
\[ \]
\vspace{-1.8in}
\begin{center}
{\mytitle} \\

\vspace{.5in}

\singlespace{
\myname \\
San Francisco, California \\  
\thisyear \\
}
\end{center}

\vspace{.5in}

%\doublespacing{\noindent
\onehalfspacing{\noindent
Sensor Networks are becoming important to different scientific and industrial
communities because of what they produce: the raw data of diverse domains. In
order to make use of the collected data, researchers may have to dissect the
characteristics of the sensor network in question, regarding different
properties such as the purpose and location of the observed data, as well as
how the data is described. For this reason, this work's first contribution is
a set of data persistence taxonomies based on the state of the art of Data
Persistence for Sensor Networks, which can be used to classify the properties
of the produced raw data. In order to evaluate the proposed taxonomies, a data
persistence component for NetBEAMS was designed and implemented. NetBEAMS is a
component-based sensor network infrastructure developed to improve the
operation of the SF-BEAMS environmental sensor network. SF-BEAMS is deployed
in Tiburon, California, and managed by the Romberg Tiburon Center (RTC). Based
on an empirical analysis regarding proposed taxonomies, a Key-Value Data Model
is proposed as an alternative to the Relational Data Model traditionally used.
Furthermore, the mongoDB database, a schema-less document-oriented database,
was selected for evaluation. Results based on the experiments suggest a novel
approach to provide External or Data-Centric persistence for networks.
Similarly, the literature supports programming languages as a better
abstraction when it comes to data access and modification by non-technical
users such as biologists. Finally, this report lends itself to future work in
the area of data persistence in sensor networks.} \vspace*{\fill}

\textbf{Keywords:} Environmental Sensor Network, Data Persistence, Taxonomies,
KVP Data Model, Document-Oriented Databases, KVP Databases, SF-BEAMS, NetBEAMS
\hspace*{\fill}
\noindent
I certify that the Abstract is a correct representation of the content of this
report.

\vspace{.5in} 

\hrule{\hspace{3.75in}} \\[-10pt]
Chair, Thesis Committee 
\hspace{2.5in}
Date
\onehalfspacing
\newpage
\[ \]
\vspace{-1.8in}
\begin{center}{ACKNOWLEDGMENTS}\end{center}

\vspace{.5in}

\noindent
\textit{``The brick walls are not there to keep us out. The brick walls are there
to give us a chance to show how badly we want something. Because the brick walls
are there to stop the people who don't want it badly enough.''} Dr. Randy
Pausch

\begin{quote}
\noindent

As Dr. Randy Pausch's last lecture reminded me after the first quarter of the
program, there are purposes for the difficult things in life as the brick walls
will always be standing one after the other in course of your life. As my
childhood dream was to become a scientist and study in America, I can say that
I might have crossed a good percentage of those brick walls. This program gave
me the wings necessary to keep dreaming and believing that anything is
possible. You've just got to dream.

I have been fortunate in these twenty-four months to meet many people who have
given me more of their personal and professional time, their companionship,
their help, and above all, their patience. They all contributed, directly or
indirectly, for the success of this work, which I kindly refer to as one of my
childhood dreams.

The academic professors were fundamental in different moments during this
program. First, I would like to thank my adviser, Dr. Arno Puder, who not only
had to put up with the times my laptop PC broke and I delayed the delivery of
components, but also the times I was just late for the meetings. He was the 
first one who believed on my work in the program and accepted me in the
team. A big thanks to him, someone whose expertise and professionalism I have
learned to admired. Dr. Dragutin Petkovic also played an important role
throughout my life during the program, advising me about work schedule,
interviews, etc. Finally, Prof. Murphy offered the hardest classes of the
program, and I must say that my writing skills were improved because of her
class.

There are four fellow researchers whom I would like to specifically thank:
Kleber Sales, who have developed the core of NetBEAMS, and for many times
explained and helped me during development. Many were the afternoons and weeks
of revisions to have a working demo. Teresa Johnson was also important for the
development of NetBEAMS and provided her feedbacks. William Murad helped me a
lot during the classes of Biology and Algorithms, while Ivan Weiss was a great
co-worker during the Operating Systems class developing TOS as one of the best
teammates during school time.

A special word of gratitude goes to the ones who made everything possible.
My co-workers from CollabNet, in special for Richard, Jack, and Adam because
their great recommendations, and to Connie, Edgar, Yiping, Norah, Kathryn and
Andres (Facebook) who I had more contact with during the first and second
times I worked over there as a full-time and an Intern, respectively.
Subversion is part of my life today! Also, Eliot from 10gen for the
innumerous days and nights on the #mongodb IRC channel and mailing list,
helping me figure out the value of mongoDB. This work was only presented
because of the help of three VERY important people I believe are important to
me: Niki is the angel that appears from nowhere to salve you! I decided to
start the program because I was very welcomed by her. During the entire
program she was always there for each of us, I am sure, and now towards the
end she stepped up and started reading and fixing the entire chapters. Niki,
there are no words to thank you. In addition, Ivan also kindly offered his
help to review the technical chapters for the gramatical errors. Man, I'm very
lucky to have you around. And finally Kleber, who suffered with the first
drafts of this work by reading so many errors and gramatical mistakes. I thank
him a lot for the help and I will definitely have a better version for him from
now on!

Moving towards to more personal acknowledgements, I would like to execute a big
of annotated thanks towards all my family and friends. They are roommates,
friends, landlords, distant friends, that I have lived with, had been with,
etc. These showed they cared about me (at least I felt it) in
what I wanted to accomplish. My gratitude to people here in the California
goes to Rhea, Sonia, Demian and Maria, Eduardo, Paula, Amanda and Eric,
Leanna, Ivan, Julie, Diane and in special to Michelle, who believed on me
since even before being in the program. The ones in Seattle who are also worth
mentioning are Eduardo and Renata, for receiving me during the attempts to
interview with Microsoft, and Wade and Fran for their friendship and support.
When it comes to the West Coast, a lot of important people are in New York and New
Jersey. Patricia is always available to talk about family, all the guys from
the Focolare Movement, who I haven't been physically in contact, but online.
Ricardo for being the best roommate and close friend, and finally Regina from
Boston, who have belived in my potential since I first arrived in this
country. My second-degree cousin Thamires with whom I have enjoyed hanging out
in the night life of San Francisco!

\medskip

I am, for sure, particularly indebted to my parents and my brothers for their
monumental, unwavering support and encouragement during these months of
dedication. For all my anties, uncles, cousins (Kaline, Nicinho and
Fernanda), they all have truly always been there for me since even before the
program started. Without my mom and dad, or my brothers Leandro and Thiago,
none of this dream would have been even possible remotely. I love you all and
I will always remember your support during these two years of less phone calls
and visits. Some brick walls were hard to climb, some of them were even more
difficult, but none of them were impossible, because I have faith in the God
of the impossible, who helped me stand up when I fell so many times. That's
why I thank God Jesus Christ for bringing another of my oldest childhood
dreams alive.
\end{quote}

\textit{``It's not about how to achieve your dreams. It's about how to lead
your life. If you lead your life the right way, the karma will take care of
itself. The dreams will come to you."} Dr. Randy Pausch

\renewcommand{\contentsname}{\vspace{-1.7in} \begin{center} \normalsize \rm TABLE OF CONTENTS \end{center}}
\renewcommand{\listfigurename}{\vspace{-1.7in} \begin{center} \normalsize \rm LIST OF FIGURES \end{center}}
\renewcommand{\listtablename}{\vspace{-1.7in} \begin{center} \normalsize \rm LIST OF TABLES \end{center}}
\renewcommand{\cftchapfont}{\normalfont}
\renewcommand{\cftchappagefont}{\normalfont}
\renewcommand{\cftchapleader}{\cftdotfill{\cftdotsep}} % formatting commands for table of contents
\renewcommand{\cftsecfont}{\normalfont}
\renewcommand{\cftsecpagefont}{\normalfont}
\renewcommand{\cftsecleader}{\cftdotfill{\cftdotsep}}

\newpage \tableofcontents 
\newpage \listoftables % comment out if you don't use tables
\newpage \listoffigures % comment out if you don't use figures

\newpage
\pagestyle{myheadings}
\pagenumbering{arabic} 
\setcounter{page}{1}
