% preamble.tex, to be used with thesis.tex
% This contains the TeX definitions for layout, style, etc., as well as the first few pages of your thesis: title page, copyright page, approval page, abstract, acknowledgments, tables of contents, tables, and figures.
% The layout commands should give the correct margins according to the graduate division's guidelines

%%%%% TeX class and packages

\documentclass[12pt,oneside]{sfsuthesis}
\usepackage{amsthm,amsmath,amssymb,amsfonts,latexsym,graphicx,enumerate,setspace,verbatim,makeidx,subfigure, multirow, url, soul}
\usepackage[subfigure]{tocloft}
\usepackage{color}                       % For creating colored text and background
%\usepackage{hyperref}                 % For creating hyperlinks in cross
%references
% other possibly useful packages: textcomp,mathrsfs,amscd,epsfig,euscript,cancel

%%%%% Layout
% These numbers might depend on your printer. Check the margins and compare them to the Graduate Division's
% guidelines. If there's something off, try playing with the numbers...
%
% For chapters:
%     Must have a minimum of 1.5in margin on left and 1in on all other sides.  Where there are page numbers
%     (whether on top or bottom), must have one additional inch between the page number and the text, for a
%     total of 2in between the edge of the paper and the text.
% For frontmatter pages:
%     The same margin numbers generally work, except for the Title Page, so you will notice that we use
%     some numbers for \textheight and \footskip right here, and then change them below, right after
%     generating the Title Page.

\hoffset=.5in 
\oddsidemargin=0in
\evensidemargin=0in
\topmargin=0.1in 
\headheight=0in
\headsep=1in
\footskip=.8in
\textwidth=5.9in
\textheight=7in

\setcounter{secnumdepth}{3}
\setcounter{tocdepth}{3}

\pagestyle{plain}

\doublespacing

%%%%% Style of theorems, definitions, examples, equations, etc.

\theoremstyle{plain} % Heading is bold, text italic.
\newtheorem{theorem}{Theorem}[chapter]
\newtheorem{lemma}[theorem]{Lemma}
\newtheorem{proposition}[theorem]{Proposition}
\newtheorem{corollary}[theorem]{Corollary}
\newtheorem{conjecture}{Conjecture}[chapter]

\theoremstyle{definition}  % Heading is bold, text is roman
\newtheorem{definition}{Definition}[chapter]
\newtheorem{example}{Example}[chapter]

\theoremstyle{remark}  % Heading is italic, text is roman
\newtheorem*{remark}{Remark}
\newtheorem*{note}{Note}
\newtheorem{claim}{Claim}[chapter]

%%%%% Appendix style

\renewcommand\appendix[1]{
\chapter*{#1}
\addcontentsline{toc}{chapter}{#1}
}

%%%%% Source-Code listings

\usepackage{courier}
\usepackage{color}
\usepackage{xcolor}
\usepackage{listings}

\lstset{commentstyle=\tiny,captionpos=t,tabsize=2,frame=lines,keywordstyle=\color{blue}\tiny,commentstyle=\color{gray}\tiny,stringstyle=\color{red}\tiny,numbers=left,numberstyle=\tiny,breaklines=true,showstringspaces=false,basicstyle=\tiny,emph={label}}

%\definecolor{darkgreen}{named}{green}
%\definecolor{darkblue}{named}{blue}
%\definecolor{darkred}{named}{red}
%\definecolor{grau}{named}{gray}
%\let\Righttorque\relax
%\lstset{
%commentstyle=\itshape\color{darkgreen},
%keywordstyle=\bfseries\color{darkblue},
%stringstyle=\color{darkred},
%extendedchars=true,
%basicstyle=\scriptsize\ttfamily,
%basicstyle=\tiny\ttfamily,
%tabsize=2,
% keywordstyle=\textbf,
%commentstyle=\color{grau},
% stringstyle=\textit,
%numbers=left,
%numberstyle=\tiny,
% für schönen Zeilenumbruch
%breakautoindent = true,
%breakindent = 2em,
%breaklines = true,
%postbreak = ,
%prebreak = \raisebox{-.8ex}[0ex][0ex]{\ensuremath{\lrcorner}},
%prebreak = \raisebox{-.8ex}[0ex][0ex]{\Righttorque},
%showspaces=false, % Keine Leerzeichensymbole
%showtabs=false, % Keine Tabsymbole
%showstringspaces=false,% Leerzeichen in Strings
%}

\usepackage{caption}
\DeclareCaptionFont{white}{\color{white}}
\DeclareCaptionFormat{listing}{\colorbox{gray}{\parbox{\textwidth}{#1#2#3}}}
\captionsetup[lstlisting]{format=listing,labelfont=white,textfont=white}

%% Use fancy chapter headers, with Jos Dingjan's modifications,
%% plus my own tweaks. This style is not part of teTeX,
%% so we are using a local (and renamed) copy.
\usepackage[Lenny]{fncychap}

%Hifens
\usepackage[latin1]{inputenc}
\usepackage[USenglish]{babel}

%Harvard citation style
\usepackage{natbib}
%enhanced items
% http://dante.ctan.org/tex-archive/macros/latex/contrib/enumitem/enumitem.pdf
\usepackage{enumitem}

\begin{document}

\pagenumbering{roman}
\thispagestyle{empty}

\[ \]
\vspace{-1.8in}

\begin{center}
{\mytitle}

\vspace{1.4in}

\singlespace{A thesis presented to the faculty of\\
San Francisco State University\\
In partial fulfillment of\\
The requirements for\\ The degree}

\vspace{.5in}

\singlespace{Master of Science\\ In\\ Computer Science}

\vspace*{\fill}

{by \\[12pt] 
\myname \\[12pt]
San Francisco, California\\[12pt]
\thismonth
\thisyear}
\end{center}

\newpage
\thispagestyle{empty}

$\mbox{}$
\vspace{3in}
\begin{center}
\singlespace{
    Copyright \copyright{} \thisyear by \myname

    \bigskip

    All rights reserved. No part of the material protected by this
    copyright notice may be reproduced or utilized in any form or by any
    means, electronic or mechanical, including photocopying, recording or
    by any information storage and retrieval system, without the prior
    permission of the author.
}
\end{center}

\newpage
\thispagestyle{empty}
\[ \]
\vspace{-1.8in}
\begin{center}
{CERTIFICATION OF APPROVAL}
\end{center}
\vspace{.5in}
\begin{quote}
I certify that I have read {\it \mytitle} by \myname, and that in my opinion this work meets the criteria for approving a thesis submitted in partial fulfillment of the requirements for the degree: Master of Science in Computer Science at San Francisco State University.
\end{quote}

\vspace{1.5in}

\hspace*{\fill}\parbox{3.5in}{
\singlespace{

\hrule{\hspace{3.5in}} \\ 
Prof. Arno Puder, Ph.D.\\
Professor of Computer Science\\
\vspace{1in}
\hrule{\hspace{3.5in}} \\
Prof. Marguerite C. Murphy, Ph.D.\\
Professor of Computer Science

}
}

\newpage
\thispagestyle{empty}
\[ \]
\vspace{-1.8in}
\begin{center}
{\mytitle} \\

\vspace{.5in}

\singlespace{
\myname \\
San Francisco, California \\  
\thisyear \\
}

\end{center}

\vspace{.5in}

%\doublespacing{\noindent
\onehalfspacing{\noindent      
Sensor Networks are becoming more and more important to the different
sciences communities bacause it provides the raw data for the different studies
related to a given expertise area. In order to make use of data, sensors are
directly or indirectly interrogated by using secondary storage devices, where
they are stored using different data models divided by schema-dependent or
schema-less models. In this way, this work summarizes the current state of the
art in data persistence for sensor networks by using taxonomies, and conducts an
empirical evaluation for a technology to be used in experiments. Moreover, this
work shows a novice use of a Key-Value Pair database system by an
implementation of a software component that is easily attached to an existing
sensor network platform. When it comes scalability, this work suggests the use
of Horizontal database partitioning in order to provide a Data-Centric
storage approach to persist data collected from an existing real-world sensor
network. Finally, it also contributes with the necessary background
documentation of the software used, among others.

}
\vspace*{\fill}

\hspace*{\fill}

\noindent
I certify that the Abstract is a correct representation of the content of this thesis.

\vspace{.5in} 

\hrule{\hspace{3.75in}} \\[-10pt]
Chair, Thesis Committee 
\hspace{2.5in}
Date
\doublespacing
\newpage
\[ \]
\vspace{-1.8in}
\begin{center}{ACKNOWLEDGMENTS}\end{center}

\vspace{.5in}
\begin{quote}
\noindent
Acknowledgments
\end{quote}

\renewcommand{\contentsname}{\vspace{-1.7in} \begin{center} \normalsize \rm TABLE OF CONTENTS \end{center}}
\renewcommand{\listfigurename}{\vspace{-1.7in} \begin{center} \normalsize \rm LIST OF FIGURES \end{center}}
\renewcommand{\listtablename}{\vspace{-1.7in} \begin{center} \normalsize \rm LIST OF TABLES \end{center}}
\renewcommand{\cftchapfont}{\normalfont}
\renewcommand{\cftchappagefont}{\normalfont}
\renewcommand{\cftchapleader}{\cftdotfill{\cftdotsep}} % formatting commands for table of contents
\renewcommand{\cftsecfont}{\normalfont}
\renewcommand{\cftsecpagefont}{\normalfont}
\renewcommand{\cftsecleader}{\cftdotfill{\cftdotsep}}

\newpage \tableofcontents 
\newpage \listoftables % comment out if you don't use tables
\newpage \listoffigures % comment out if you don't use figures

\newpage
\pagestyle{myheadings}
\pagenumbering{arabic} 
\setcounter{page}{1}
